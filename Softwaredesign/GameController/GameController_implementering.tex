\documentclass[Softwaredesign/Softwaredesign_main.tex]{subfiles}
\begin{document}
\section{Implementering af GameController klasse på PSoC}\label{sec:GameController_implementering_bilag}
Implementeringen af GameController klassen blev gjort i PSoC creator, hvor sproget C er brugt. Da C ikke understøtter klasser er der bare lavet en header fil med alle funktionernes prototyper, der er dog ikke instantieret nogle variabler i denne klasse. Dette er gjort i den tilhørende C-fil, hvor alle variabler i klassen er gjort static. Dette er gjort for at prøve at efterligne c++, som har private variabler. Static gør at det kun er i c-filen for GameController klassen, hvor de forskellige variabler kan tilgås. Hvis man ønsker at tilgå variablerne i andre filer skal det gøres gennem funktionerne i klassen. Grunden til at sproget c er valgt er af den simple grund at vi ville bruge en PSoC til hver af vores playerside. En PSoC skal programmeres ved hjælp af PSoC creator, der gør det let for brugeren at bruge deres katalog af komponenter, som er meget lette at bruge, da både tilhørende funktioner kan bruges(for eksempel at starte og stoppe en timer er der to funktioner) og ved hjælp af den GUI, der er givet til at konfigurere komponenter. PSoC creator understøtter kun c og ikke c++. Normalt bruges C også til embeddede systemer af PSoC's størrelse, da det ofte fylder mindre end c++ kode.

\end{document}