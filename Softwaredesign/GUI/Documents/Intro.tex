\documentclass[Softwaredesign/Softwaredesign_main.tex]{subfiles}

\begin{document}

\textbf{Display}\\
I dette afsnit beskrives udviklingen af "Display" delen af RPiApp'en: herfra refereret til som \textit{GUI}. Der beskrives på hvilket framework GUI'en er baseret, samt brugen af udviklings værktøjet. Der bliver en kort gennemgang af selve udseendet af GUI'en.  Der bliver dog ikke talt om design "sproget", i forhold til grafisk design og æstetik. 

\textbf{Overvejelser}\\
Når der skal udvikles en GUI, skal der (næsten) altid vælges et framework at arbejde ud fra. I dette tilfælde er der en klar grænse for valg af framework: Det skal kunne fungerer på et Linux baseret systemet. Derfor vælges der QT creator som framework for udvikling af GUI. Det er kryds-platform kompatible: Windows, Linux og MacOS. Desuden har det den fordel at det også er hovedsageligt  C++ baseret. 

\textbf{Framework}\\
Frameworket som GUI'en er udviklet på QT creator. Hvilket er et udviklings værktøj bestående både af en grafisk del, samt en kode del. 

\end{document}