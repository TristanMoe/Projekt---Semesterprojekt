\documentclass[Softwaredesign/Softwaredesign_main.tex]{subfiles}

\begin{document}

\textbf{Display}
I dette afsnit beskrives udviklingen af "Display" delen af RPiApp'en: Her fra refereret til som \textit{GUI}. Der beskrives på hvilket framework GUI'en er baseret, samt brugen af udviklings værktøjet. Der bliver en kort gennemgang af selve udsenet af GUI'en.  er bliver dog ikke talt om design "sproget", i forhold til grafisk design og æstetik. Der bliver en længere gennemgang af hvordan GUI'en fungere på et software niveau. Både i forhold til kode, men også i forhold til layout.

\textbf{Overvejelser}
Når der skal udvikles en GUI, skal der (næsten) altid vælges et framwework at arbejde ud fra. I dette tilfælde er der en klart grænse for valg af platform: Det skal kunne fungere på et Linux baseret systemet. Derfor vælges der QT creator som framework for udvikling af GUI. Det er kryds-platform kompatible: Windows, Linux og MacOS. Desuden har det den fordel at det også hovedsageligt er C++ baseret. 

\textbf{Framework}
Frameworket som GUI'en er udviklet på QT creator frameworket. Hvilket er et udviklings værktøj bestående både af en grafisk del, samt en kode del. Den grafiske del består af valg af layout; layout kunne være hvor billeder er, hvad baggrunden er, og meget mere. For at gør de her billeder, tekst, baggrund osv. dynamiske bruges koden del så. Så hvis et billede skulle skifte fra en rød kop til en grøn kop, eller fra en grøn kop til slet intet billedet gøres det gennem koden, men billedets placering er defineret ud fra den grafiske del. 



\end{document}