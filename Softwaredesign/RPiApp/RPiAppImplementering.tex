\documentclass[Softwaredesign/Softwaredesign_main.tex]{subfiles}
\lstset{language=C++,
                basicstyle=\ttfamily,
                keywordstyle=\color{blue}\ttfamily,
                stringstyle=\color{red}\ttfamily,
                commentstyle=\color{green}\ttfamily,
                morecomment=[l][\color{magenta}]{\#},
                showstringspaces=false,
                breaklines=true,
                emph={std, cout, endl, pthread_t},
                emphstyle={\color{purple}}
}
\begin{document}
\section{Implementering af RPiApp}
Softwareapplikation RPiApp er generelt udviklet objektorienteret i sproget C++17. Sproget bruges da det kan kompileres og afvikles på den valgte computerenhed, Rasberry Pi W Zero. C++ bruges til at udvikle det event driven arkitektur og det trådsikret system, men der bruges også elementer af C11 til at få adgang til filoperationer i Kernal Space. Implementering tager udgangspunkt i figur \ref{swdesign:fig:cd_RPiApp_}, som viser klassediagrammet for RPiApp. Klassediagrammet viser de forskellige attributer, operationer og relationer mellem klasserne. \\
Alt kode er ikke forklaret for de forskellige klasser, her henvises til den fulde source-kode i bilaget \textbf{Kildekode}. 

\subsection{GameController}
Den første klasse som implementeres er klassen GameController, som er den logiske controller i softwaresystemet. Som nævnt i designafsnittet, er softwaresystemet designet ud fra event driven arkitektur, MsgQueue-systemet, dispatchers og handlers. \\
GameController klassen følger stadie transitionen som vist i figur \ref{arch:fig:stm_Game} i softwarearkitekturen. 
\begin{lstlisting}[caption={Enumeration stadier for GameController}, label=list:Game_State]
enum GameControllerState
{
		GAME_STATE_IDLE,
		GAME_STATE_STARTING,
		GAME_STATE_PLAYING,
		GAME_STATE_ENDGAME,
		GAME_STATE_SERVICE,
};
GameControllerState state = GAME_STATE_IDLE;
\end{lstlisting}
GameControllers stadie er afgørende for hvilke beskeder som modtages og afvikles, og hvilke forkastes i dispatcher'en. Dispatcher'en består af to nested switches, hvor hovedswitchen er til at kontrollere at de korrekte beskeder afvikles for stadierne, og den anden er til at håndtere MsgId'erne - MsgId'erne er defineret i MsgProtocol klassen: 
\begin{lstlisting}[caption={Dispatcher for GameController}, label=list:Disp_Game]
Void GameController::dispatcherGameMessage(unsigned long msgID, Message * msg)
{
    switch (state)
	{
	case GAME_STATE_IDLE:
	{
		switch (msgID)
		{
		case SYSTEM_STATE::IDLE:
		{
			cout << "SYSTEM_STATE::IDLE" << endl;
			handleIdleRequest();
		}
		break;

		case BallDispenser::NEW_INFO:
			cout << "NEW_INFO : ";
			handleBallDispenserResponse(static_cast<BallDispenserResponse *>(msg));
		break;

		case WebPageServer::ID_INFO_READY :
		handleWebPageResponse(static_cast<WebPageResponse *>(msg));
		break;
		}
	}
	break;
	
	... Resten af cases ... 
\end{lstlisting}
\textbf{MsgId og Message * msg-instanser}\\
Hvert defineret MsgID i MsgProtocol har sin egen handler. Da der bruges static\_cast til at allokerer resourcer til en nedarvet instanse af klassen Message, er det vital, at der ikke allokeres resourcer til en forkert 'msg' instans. Her bruges MsgId'erne til at associerer en nedarvet instans. Nedenfor ses et eksempel på respons fra Playerside (Data læst): 
\begin{lstlisting}[language=C++, caption={Håndtering af PlayersideResponse}, label=list:PS_Response]
void GameController::handlePlayersideResponse(PlayersideResponse * msg)
{
	cout << "HandlePlayersideResonse" << endl;

	if(msg->playerside_ == 1)
	{
	team1_.setCups(msg->readPtr_);
    }

	else if(msg->playerside_ == 2)
	{
	team2_.setCups(msg->readPtr_);
	}
    DisplayObj_->DisplayMsgQueue_.send(Display::CUPS_UPDATED);

	cout << "TEAM1: "<< team1_.getCups() << endl;
	cout << "TEAM2: "<< team2_.getCups() << endl;

	if(team1_.getCups().all() && team2_.getCups().all() && (this->state == GAME_STATE_STARTING))
	{
		if(WebPage_Information){
			GameControllerMsgQueue_->send(WebPageServer::ID_INFO_READY);
		}
	}
	else if((team1_.getCups().none() || team2_.getCups().none()) && (this->state == GAME_STATE_PLAYING))
	{
		this->state = GAME_STATE_ENDGAME;
		if(team1_.getCups().none())
			GameControllerMsgQueue_->send(SYSTEM_STATE::ENDGAME, new EndGameCfm(1));

		else if(team2_.getCups().none())
			GameControllerMsgQueue_->send(SYSTEM_STATE::ENDGAME, new EndGameCfm(2));
	}
}
\end{lstlisting}
Den response som modtages er lavet som en nedarvet klasse af baseklassen Message. Klassen PlayersideResponse indeholder en readPtr, som er det data der er læst fra Playerside enheden: 
\begin{lstlisting}[caption={PlayersideResponse klasse}, label=list:Response_Class]
class PlayersideResponse : public Message
{
public:
	PlayersideResponse(const char * readPtr = nullptr, int playerside = 0)
	: readPtr_(readPtr), playerside_(playerside)
	{
	}
	const char * readPtr_ = nullptr;
	const int playerside_ = 0;
};
\end{lstlisting}
Alle beskeder modtaget gennem GameControllers MsgQueue behandles tilsvarende som listing \ref{list:PS_Response}. 

\subsection{I2C}
Klassen I2C er baseklassen for alt i2c kommunikation. Her implementeres alle funktioner, som både Playerside og BallDispenser klasserne skal bruge - hovedsageligt filoperationerne tilhørende write- og readfunktionerne. BallDispenserRead og -Write, samt PlayersideRead og -Write nedarver fra I2C, men åbner selv filen de skal opererer på og bruger den nedarvede filpointer. Klassens read- og writefunktioner er inspireret af slides fra HAL-undervisning \autocite{HAL2018}\\
Se klassens struktur i listing \ref{list:i2c_class}
\begin{lstlisting}[caption={I2C headerfil}, label=list:i2c_class]
class I2C
{
public:
  I2C(){};
  virtual void translateMessage() {}
  int readI2C();
  const char * getReadBuffer();

  int writeI2C(const char * command, int length);

  virtual ~I2C(){}

  int filePointer_;

protected:
  char readBuffer_[8] = {0};
  char writeBuffer_[8] = {0};
};
\end{lstlisting}
Der benyttes også polymorphism i form af funktion translateMessage, som skal implemteres i PlayersideRead og BallDispenserRead. 

\subsection{Playerside og BallDispenser (Read og Write)}
De tidligere Playerside og BallDispenser boundary klasser er opdelt i funktionalitet; read og write. I begge tilfælde nedarves fra I2C klassen, og i constructoren åbnes den fil, som er prædefineret i device træet for driveren (filepointer data memberen er fra I2C klassen). Klassenernes constructor er inspireret af slides fra HAL-undervisning \autocite{HAL2018}\\
Nedenfor ses eksemplet for PlayersideRead, men det er samme princip for de andre klasser: 
\begin{lstlisting}[caption={PlayersideRead constructor}, label=list:construct]
PlayersideRead::PlayersideRead(const char * filename, MsgQueue * GameControllerQueue, int playerside)
	: GameControllerQueue_(GameControllerQueue), playerside_(playerside)
{
	filePointer_ = open(filename, O_RDWR);
	if (filePointer_ == -1) {
		cout << "PS" << playerside_ << "ReadOpen: Failed to open with err: " << strerror(errno) << endl << endl;
	}
}
\end{lstlisting}

\subsection{WebPage implementering}\label{sec:webpage_implementering}
I dette afsnit ses implementeringen for WebPage. Der er kun medtaget de væsentligste metoder og parametre. For yderligere detaljer henvises til kildekoden. 
\subsubsection{Client}
Client henviser til den HTML-fil, som afvikles i web browseren. Listing \ref{list:webpage_interface} viser hvordan hjemmesidens layout er implementeret, herunder tekstfelter, track bars og Start-knap.
\begin{lstlisting}[caption={WebPage interface layout}, label=list:webpage_interface]
<!DOCTYPE html>
  <html>
  <meta charset="utf-8" />
  <head>
  <title>Beer Pong Game</title>
  <style type="text/css">
  #wrap {
    width: 805px;
    height:575px;
  }

  .left{
    width:400px;
    height:200px;
    float:left;
  }
  
  #display1{
    width: 28%;
    height: 100px;
    background: #ff0000;
    transition: background 100ms;
    margin-top: 30px;
    border: 1px solid #000;
  }
  
  //similar for .right and #display2
  ...
  
<body>
<h1>Beer Pong</h1>
<div id="wrap">
  <div class="left">
  <h3>Team 1</h3>
  <p> Enter username 1: </p>
  <input type="text" id="username1" name="username1" maxlength="15"/>
  <p> Enter username 2: </p>
  <input type="text" id="username2" name="username2" maxlength="15"/>
  <p> Enter teamname: </p>
  <input type="text" id="teamname1" name="teamname1" maxlength="15"/>
  <br />
  <p> Enter team color: </p>
  <div class="picker1">
  Red <input type="range" min="0" max="255" step="1" id="red1" value="255">
  Green <input type="range" min="0" max="255" step="1" id="green1" value="0">
  Blue <input type="range" min="0" max="255" step="1" id="blue1" value="0">
  <div id="display1"></div>
</div></div>
 
//Similar for team2 

<button type="button" onclick="acquireInfo()">Start</button>
\end{lstlisting}

Listing \ref{list:webpage_handlers} viser JavaScript implementeringen af de event handlers, som er associeret med WebSocket objektet. Et for loop med event listeners sørger for at opdatere de viste farver, når disse justeres. Funktionen acquireInfo() kaldes, når der trykkes på 'Start'-knappen, og sørger for at konkatenere alle brugerinputs til én tekststreng, som sendes til Server. 

\begin{lstlisting}[caption={WebPage event handlers}, label=list:webpage_handlers]

<script language="javascript" type="text/javascript">

var wsUri = "ws://10.9.8.2:3000/";

function init(){
    output = document.getElementById("output");
    testWebSocket();
}

function onOpen(evt){
    writeToScreen("CONNECTED");
}

function onClose(evt){
    writeToScreen("DISCONNECTED");
}

function onMessage(evt){
    writeToScreen('<span style="color: blue;">RESPONSE: ' + evt.data+'</span>');
}

function onError(evt){
    writeToScreen('<span style="color: red;">ERROR:</span>');
}

function doSend(message){
    writeToScreen("SENT: UserInfo");
    websocket.send(message);
}

// Displaying team colors
var input1 = document.querySelectorAll('input[type="range"]');
for(var i = 0; i<input1.length; i++) {
    input1[i].addEventListener("input", function(){
    red1 = document.getElementById("red1").value,
    green1 = document.getElementById("green1").value,
    blue1 = document.getElementById("blue1").value;

    var display1 = document.getElementById("display1");
    display1.style.background = "rgb(" + red1 + ", " + green1 + ", " + blue1 + ")";
    });
}
...
//Similar for team2

function acquireInfo(){
    username1 = document.getElementById("username1").value;
    username2 = document.getElementById("username2").value;
    ...
    blue2 = document.getElementById("blue2").value;

    gameInformation = username1+'-'+username2+'-'+teamname1+'-'+red1+'-'+green1+'-'+blue1+'-'+username3+'-'+username4+'-'+teamname2+'-'+red2+'-'+green2+'-'+blue2+'#';
    doSend(gameInformation);
}
</script>
\end{lstlisting}

\subsubsection{Server}
Server er implementeret som en C++ klasse kaldet WebPAge. Implementering af funktionen WSInit() fremgår af listing \ref{list:WSInit}. OnMessage funktionen kaldes for et Hub objekt h, som er en del af uWS biblioteket. Det fremgår desuden, at onMessage er implementeret som et lambda udtryk. Det betyder, at den skrives som en inline-funktion, som kan udføres umiddelbart efter, at den er defineret. Capture listen [\&] tillader, at den kan tilgå alle variabler i det scope, den er omgivet af. I et for loop indlæses tekststrengen i en buffer. Herefter anvendes getline funktionen i et for loop til at splitte strengen ad og gemme dem i en STL Vector. Set-metoder fra UserInfo klasse anvendes til at sætte members og disse sendes i en dynamisk allokeret besked kaldet WebPageResponse til GameControllers MsgQueue. Endelig sendes en response besked til Client.
\begin{lstlisting}[caption={Initering af WebSocket}, label=list:WSInit]
void WebPage::WSInit()
{
	h.onMessage([&](uWS::WebSocket<uWS::SERVER> *ws, char *message, size_t length, uWS::OpCode opCode) {

	// Splitting string and storing substrings in vector
	ostringstream buffer;
	for (int i = 0; message[i] != '#'; i++)
	buffer << message[i];

	string gameInformation = buffer.str();

	char delimiter = '-';
	vector<string> tokens;

	istringstream str(gameInformation);
	for (string substring; getline(str, substring, delimiter); tokens.push_back(substring));

	//Setting team2
	team2_.setBlue(tokens.back());
	tokens.pop_back();
	team2_.setGreen(tokens.back());
	tokens.pop_back();
	team2_.setRed(tokens.back());
	tokens.pop_back();
	team2_.setTeam(tokens.back());
	tokens.pop_back();
	team2_.setUser2(tokens.back());
	tokens.pop_back();
	team2_.setUser1(tokens.back());
	tokens.pop_back();

     ...
        
     //Similar for team1
        
	// Sending to GameController class
	GameControllerMsgQueue_->send(ID_INFO_READY, new WebPageResponse(team1_, team2_));

	// Sending back to client
	ws->send(message, length, opCode);
});}
\end{lstlisting}

\begin{lstlisting}[caption={Run() funktion for WebPage}, label=list:webpage_run]
void WebPage::run()
{
	WSInit();
	while(true)
	{
	    if (h.listen(3000)) {
			h.run();
		}
	}
}
\end{lstlisting}

\subsection{Display Implementering}
\subfile{Softwaredesign/GUI/GUI_implementering.tex}\label{sec:GUI_implementering_doc}


\end{document}