\documentclass[Modultest/Modultest_main.tex]{subfiles}
\begin{document}

\lstdefinestyle{customc}{
    breaklines=true,
    language=C,
    showstringspaces=false,
}
\lstset{style=customc}

\section{WebPage Modultest}
I dette afsnit beskrives modultesten for WebPage. WebPage er et modul, som har til opgave at modtage brugerinputs i form af hold- og brugernavne samt holdfarver. Disse informationer indtastes på smartphone af brugeren, og herfra skal de behandles og videresendes til GameController klassen. For at dette er muligt, er det nødvendigt at kommunikationen mellem client og server forløber korrekt, og at WebSocket forbindelsen er pålidelig.\\I de følgende tests tages der udgangspunkt i et simpelt testprogram, hvor server udskriver holdnavne, brugernavne og holdfarver hver gang disse sendes fra hjemmesiden (ved tryk på 'Start'). I ikke-funktionelle krav for WebPage er det specificeret i krav K5.8, at der højst må ske en fejl i afsendingen 1 ud af 50 gange. Dette testes i det følgende ved at indtaste spil oplysninger på hjemmesiden og sende dem, hvorefter terminal output undersøges. Af praktiske årsager foretages der en stikprøve på n=20, dvs. førnævnte test gentages 20 gange, mens navne og farver varieres.
\\Når dette skal testes vil resultatet være binært, forstået på den måde, at enten sendes data korrekt til server og stemmer overens med det indtastede, eller også gør det ikke. Hvert forsøg på at sende har to mulige udfald: succes og fiasko og det antages at data er binomialfordelte. 



\end{document}