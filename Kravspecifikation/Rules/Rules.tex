\documentclass[Kravspecifikation/Kravspec_Main.tex]{subfiles}
\begin{document}
\section{Beer Pong regler} \label{sec:rules}
Systemet bygger på spillet Beer Pong. Der findes mange forskellige regler for dette spil. Nogle regler er svære at håndtere for systemet, og det er endnu sværere at håndtere forskellige regler samtidig. Derfor defineres nu de regler som systemet udvikles efter. Vigtige ord er fremhævet med fed kursiv. Disse ord vil blive brugt andre steder og derfor vigtige at forstå. Nogle af disse begreber vil blive uddybet for at passe til systemet.

Før spillets start, skal spillet gøres klar til brug (\textit{\textbf{opstart}}). Der skal placeres 6 kopper i en trekantet form i hver ende af bordet. Disse kopper skal fyldes med øl (skal ikke fyldet helt op). Der skal også anskaffes 2 bordtennis bolde.  Der er i spillet to \textit{\textbf{hold}} bestående af 2 personer hver. De to hold står i hver deres ende af bordet med hver deres 6 kopper. Der besluttes hvilket \textit{\textbf{hold}} der starter med at have \textit{\textbf{turen}} og dermed er det \textit{\textbf{aktive hold}}, og det andet hold er \textit{\textbf{modstander holdet}}.  Når \textit{\textbf{opstarten}} er færdig, kan spillet starte og hermed er \textit{\textbf{spillet i gang}}

Det \textit{\textbf{aktive hold}} kaster en bordtennisbold hver, en af gangen. Hvis en bold rammer i en kop, skal \textit{\textbf{modstander holdet}} drikke indholdet i denne kop, og fjerne koppen fra trekanten af kopper. Hvis begge bolde rammer i to forskellige kopper, skal \textit{\textbf{modstander holdet}} drikke indholdet af begge kopper. Hvis begge bolde rammer i én kop, skal \textit{\textbf{modstander holdet}} drikke indholdet af den ramte kop, samt indholdet af to andre kopper (i alt 3 kopper). Hvis ingen af boldene rammer nogen kop, skal der ikke drikkes noget.

Når begge bolde er kastet og der er drukket indholdet af de nødvendige kopper, skifter \textit{\textbf{turen}}. Dvs. \textit{\textbf{modstander holdet}} bliver nu det \textit{\textbf{aktive hold}} og er i besiddelse af de to bolde. Det tidligere \textit{\textbf{aktive hold}} bliver til \textit{\textbf{modstander holdet}}. Spillet fortsætter nu som tidligere beskrevet.

\textit{\textbf{Spillet er slut}} når alle kopper er fjernet fra den ene ende af bordet. Dette kan ske på flere måder: Der er én kop tilbage og den rammes, der er to kopper tilbage og begge koppe rammes (eller en kop rammes to gange) i den samme tur eller der er tre kopper tilbage og begge bolde rammer i den samme kop, i den samme tur. \textit{\textbf{Taber holdet}} er det hold som har fået fjernet alle sine kopper i sin ende af bordet. \textit{\textbf{Vinder holdet}} er det andet hold. 

\end{document}