\documentclass[HardwareDesign/HardwareDesign_main.tex]{subfiles}
\begin{document}
\section{Cup Sensor}
I dette afsnit beskrives udviklingen af en sensor (Cup Sensor blokken) der kan registrere når en kop placeres på en bestemt placering på bordet der udvikles i dette projekt. Sensoren skal hvis muligt også detektere når en bold rammer i koppen (er 
'should' i MoSCoW analysen). Dette kan gøres på flere forskellige måder som undersøges i det følgende underafsnit. Derudover skal der laves noget hardware der gør det muligt at få data fra 6 sensorer 'samtidig' (en del af Cup Holder Controller blokken)

\subsection{Teknologiundersøgelse}
For at detektere en kop der er placeret, skal koppen med øl påvirke nogle fysiske egenskaber for en sensor, som dermed kan detekteres. Der overvejes forskellige egenskaber som kan detekteres, som kan ses i listen nedenfor. De forskellige punkter gennemgås detaljeret bagefter.
\begin{enumerate}
    \item \label{itm:cupSensor_weight} Kraft.
    \item \label{itm:cupSensor_lightReflection} Lys refleksion. 
    \item \label{itm:cupSensor_lightAbsorbtion} Lys absorption.
    \item \label{itm:cupSensor_capacitive} Dielektrikum.
\end{enumerate}

En kop med øl vil have en større masse end den luft den erstatter, derfor vil den overflade koppen står på blive udsat for en større kraft end før. Denne ændring kan dermed detekteres. En helt simpel metode vil være vha. en kontakt. Dette kan dog ikke bruges til at detektere at en bold rammer i koppen. En bold i koppen vil øge den samlede masse af koppen. En bold vil dog ikke veje så meget sammenlignet med en kop med øl. En bold i bevægelse der lander i koppen vil dog blive udsat for en kraft i det den skal deaccelereres, den overflade koppen står på vil derfor blive udsat for den samme kraft i modsat retning. Hvis der derfor laves en vægt-/kraftsensor kan denne muligvis detektere både placering af kop, men også detektere at en bold rammer i. Begge disse metoder vil kræve en bevægelse af den overflade som koppen står på. Det er et krav til at systemet skal være vandtæt,  dette er svært at overholde hvis der er ting som skal bevæge sig. Det vil sikkert være muligt, men formålet med dette projekt er ikke at arbejde med materialer mm.

Grundet problemet med at holde systemet vandtæt, overvejes nu kun metoder som kan "se"  igennem bordoverfladen. En kop reflekterer lys. Dette kan derfor være muligt at sende lys op gennem en gennemsigtigt bordoverflade, og måle på det lys der reflekteres tilbage. Dette kræver en gennemsigtig bordoverflade. Derudover vil det måske være muligt at detektere at en bold rammer i da der vil reflekteres mere lys når bolden er der.

En kop, både gennemsigtig og ikke gennemsigtig, vil absorbere/blokere for lys. Dette kan bruges på forskellige måder. Fx kan der måles på det lys som modtages fra omgivende lys. Hvis der bliver placeret en kop som dermed blokerer for/absorberer dette lys, kan det detekteres. Denne metode vil dog være afhængig af den omgivende lys, og det kan variere meget da systemet kan blive brugt i meget lyse omgivelser og i meget mørke omgivelser. Dette er derfor ikke så god en metode. En anden metode der bruger en kops absorbtionsegenskaber er at sende lys i den ene side af en kop og måle på den anden side.  Denne metode er dog ikke så elegant da det enten vil kræve neddybninger i bordet til kopperne eller at sensoren stikker op fra bordet. Begge disse situationer ønskes ikke, da der ønskes en flad bordoverflade.

En kop med øl, vil have en anden dielektrikum end luften. Derfor kan der laves en kapacitiv sensor som måler kapaciteten (som er afhængig af dieletrikum) som påvirkes af materialet omkring kop sensoren (vil blive forklaret mere i detalje). Denne metode har den fordel at man kan "se" igennem bordet, så længe overfladen har en relativ lav dielektricitetskonstant.

Der er, som beskrevet, nogle metoder hvor der er åbenlyse ulempe som vælges fra med det samme. Der vælges to metoder som undersøges nærmere: lys reflektion og dielektrikum. Disse metoder undersøges til sådan en grad så det er muligt at træffe et valg mellem de to metoder (eller fravælge dem begge og finde en ny metode). Teknologiundersøgelserne bliver hermed brugt til en form for 'proof of concept'.  

\subsubsection{Kapacitiv sensor}
Til at måle en ændring i dielektricitetskonstanten, benyttes en kapacitiv sensor. Fra den første EFYS øvelse, blev det vist at man kan male den frie kapacitet af en metalplade. Den kapacitet som måles er kapaciteten mellem pladen og en sfærisk overflade uendelig langt væk. I realiteten er det bare en stel forbindelse som er tilstrækkelig nok væk. Når der placeres en kop med øl vil der være en større dielektricitetskonstant som dermed øger den frie kapacitet. Det kan betragtes som at pladens areal nu forøges til arealet af overfladen af væsken. Der overvejes også en metode hvor der benyttes to plader, hvor der måles kapaciteten mellem de to plader se figur \ref{fig:two_plates_cap}.

Til at måle kapaciteten af en så relativ lille kondensator kan der benyttes CapSense på PSoC'en. Her skal der ikke benyttes noget ekstra hardware end det der allerede er i PSoC'en (selve microcontrolleren) samt en kondensator som Cypress kalder $C_{MOD}$ med værdien 2.2nF for en PSoC 5LP \autocite[115]{AN64846}. Den er allerede på udviklingskittet på ben P15[4] \autocite[2]{PSoCKitSchematics}. Til CapSense er der knyttet en del automatisk signal behandling. En funktionallitet der er i CapSense er at den benytter hvad de kalder en 'baseline'\autocite[18]{AN64846}. Denne baseline er den læste værdi når der ikke er en kop, eller andre ting der skal måles med CapSense. Nyttesignalet beregnes ud fra baseline så signalet er 0 (eller tæt på) når der ikke er en kop. Denne baseline kan ændre sig som omgivelserne ændre sig, fx temperaturændring. Der benyttes et lavpasfilter til at opdatere baseline, så længe den holder sig inden for en given 'noise threshold', ellers betragtes det som et nyttesignal.

Der laves først en ikke dokumenteret test hvor der undersøges de to forskellige metoder som tidligere beskrevet. Den ene er hvor den frie kapacitet måles, den anden er hvor der benyttes to plader. Til den frie kapacitet bruges en kopperplade på ca 30mmx33mm. Til den med to plader benyttes den samme plade, men med en ledning rund om den som fungere som den anden plade. Der observeres ikke nogen synlig forskel i signalet mellem de to metoder, derfor fortsættes nu kun med den frikapacitet.

Der benyttes testopstillingen som det ses på figur \ref{fig:cap_tape_testopstilling} og figur \ref{fig:cap_testopstilling}. 
Der laves forskellige test. Test hvor der placeres en kop med øl og fjernes igen. Test hvor der placeres en øl og der slippes en bold som falder ned i koppen. Det testes hvad indflydelsen af en hånd på toppen af koppen har. Der testes også med ca 110ml øl og 220ml øl. Til at få dataen fra CapSense benyttes PSoC'ens CapSense Tuner program. Målingerne kan ses på figur \ref{fig:cap_test_place_and_remove_110}, \ref{fig:cap_test_place_and_hand_110}, \ref{fig:cap_test_place_and_drop_110}, \ref{fig:cap_test_place_and_remove_220} og \ref{fig:cap_test_place_and_drop_220}


\begin{figure}[H]
    \centering
    \includegraphics[width=0.8\textwidth]{HardwareDesign/CupSensor/graphics/CapTest/tape_plate.jpg}
    \caption{Akrylplade hvor der er tapet en kopperplade på ca 30mmx33mm på undersiden.}
    \label{fig:cap_tape_testopstilling}
\end{figure}

\begin{figure}[H]
    \centering
    \includegraphics[width=0.8\textwidth]{HardwareDesign/CupSensor/graphics/CapTest/cap_testopstiling.jpg}
    \caption{Kopperplade fra \ref{fig:cap_tape_testopstilling} er nu forbundet til PSoC'en og det er vist hvor koppen vil blive placeret.}
    \label{fig:cap_testopstilling}
\end{figure}
\newpage
\textbf{Målinger ved 110ml øl}
\begin{figure}[H]
    \centering
    \includegraphics[width=\textwidth]{HardwareDesign/CupSensor/graphics/CapTest/placingAndRemovingCup1(beer).jpg}
    \caption{Test resultat ved placering (ca ved x=250) og fjernelse af kop (ca ved x=425). Øverste halvdel: rådata. Nederste halvdel: signal data, beregnet ud fra baseline}
    \label{fig:cap_test_place_and_remove_110}
\end{figure}

\begin{figure}[H]
    \centering
    \includegraphics[width=\textwidth]{HardwareDesign/CupSensor/graphics/CapTest/placingCupAndPutingHandOnTop1(beer).jpg}
    \caption{Test resultat ved placering af kop (ca ved x=200) og placering af en flad hånd på toppen af koppen (ca ved x=300) og fjernelse af hånd (ca ved x=400). Øverste halvdel: rådata. Nederste halvdel: signal data, beregnet ud fra baseline}
    \label{fig:cap_test_place_and_hand_110}
\end{figure}

\begin{figure}[H]
    \centering
    \includegraphics[width=\textwidth]{HardwareDesign/CupSensor/graphics/CapTest/placingCupAndDroppingBall1(beer).jpg}
    \caption{Test resultat ved placering af kop (ca ved x=185) og bold som rammer i koppen (ca ved x=320). Øverste halvdel: rådata. Nederste halvdel: signal data, beregnet ud fra baseline}
    \label{fig:cap_test_place_and_drop_110}
\end{figure}

\textbf{Målinger ved 220ml øl}
\begin{figure}[H]
    \centering
    \includegraphics[width=\textwidth]{HardwareDesign/CupSensor/graphics/CapTest/placingAndRemovingCup2(beer-220ml).jpg}
    \caption{Test resultat ved placering (ca ved x=175) og fjernelse af kop (ca ved x=300). Øverste halvdel: rådata. Nederste halvdel: signal data, beregnet ud fra baseline}
    \label{fig:cap_test_place_and_remove_220}
\end{figure}


\begin{figure}[H]
    \centering
    \includegraphics[width=\textwidth]{HardwareDesign/CupSensor/graphics/CapTest/placingCupAndDroppingBall1(beer-220ml).jpg}
    \caption{Test resultat ved placering af kop (ca ved x=240) og bold som rammer i koppen (ca ved x=350). Øverste halvdel: rådata. Nederste halvdel: signal data, beregnet ud fra baseline}
    \label{fig:cap_test_place_and_drop_220}
\end{figure}

\textbf{Diskussion af resultater}\\
Der ses på figur \ref{fig:cap_test_place_and_remove_110} at det er tydeligt om der er en kop eller ej. Det ses også at der kommer en spids når koppen placeres og når den fjernes, dette observeres også ved gentagne målinger. Dette kan muligvis skyldes at en hånd kommer i nærheden af sensoren. Det blev ikke målt, men det ses også at der er et relativ lavt signal-støjforhold (SNR), i forhold til den optiske sensor som undersøges i næste afsnit.

På figur \ref{fig:cap_test_place_and_hand_110} ses det der kommer en ændring i signalet når der placeres en hånd på toppen af koppen. Der aflæses fra grafen (den nederste halvdel) at signalet ca. stiger med $\frac{20}{200} = 10\%$. Der ses desuden af signalet stiger til ca. det samme niveau som spidsen ved placering af koppen, dette kan muligvis være med til at forklare at spidsen skyldes at der benyttes en hånd til at placere koppen.

På figur \ref{fig:cap_test_place_and_drop_110} ses det at der kommer en ændring i den stationære værdi når en bold rammer i koppen. Der aflæses fra grafen (den nederste halvdel af figur \ref{fig:cap_test_place_and_drop_110}) at den stationære værdi ca. stiger med $\frac{30}{180} = 17\%$. Der ses også at der kommer en spids idet bolden rammer i koppen, denne spids er ca $\frac{40}{180} = 22\%$ mere end signalet hvor der kun er en kop. Forøgelsen af signalet kan skyldes at bolden vil forøge den samlede overflade areal af øllen og dermed vil øge den frie kapacitet. Dette kan indikere at det måske er muligt at detektere en bold, da der er en ændring i signalet.

På figur \ref{fig:cap_test_place_and_remove_220} og figur \ref{fig:cap_test_place_and_drop_220} ses nogenlunde den samme for rådataen (øverste halvdel), udover at den procentvise ændring når bolden rammer i er mindre. Men det observeres også at signalet beregnet ud fra baseline (nederste halvdel af figurerne) er for stort, da det kun er 8 bit. Hvis der skal arbejdes videre med denne metode er det derfor vigtigt at få kalibreret og indstillet CapSense rigtigt.

\subsubsection{Optisk sensor}
Til at lave en optisk sensor benyttes en IR LED samt en eller flere fotodioder som er sensitive over for IR lys. Der benyttes kredsløbet som ses på \ref{fig:optic_test_diagram}. Operationsforstærkeren fungerer som en strømforstærker, som forstærker strømmen fra fotodioden og laver den om til en spænding. Spændingen ved AmpOut måles vha. Analog Discovery 2. Der er i denne undersøgelse en IR LED (SFH485\autocite{SFH485}) i centrum af koppen og og to fotodioder (SFH203FA\autocite{SFH203FA}) i siderne som det ses på figur \ref{fig:optic_opstilling}. 
\begin{figure}[H]
    \centering
    \includegraphics[width=\textwidth]{HardwareDesign/CupSensor/graphics/OpticTest/diagram.PNG}
    \caption{Test kredsløb for optisk sensor}
    \label{fig:optic_test_diagram}
\end{figure}

\begin{figure}[H]
    \centering
    \includegraphics[width=\textwidth]{HardwareDesign/CupSensor/graphics/OpticTest/Optic_testopstillign.jpg}
    \caption{Test opstilling for optisk sensor. Den lilla diode er den infrarøde LED og de to sorte er fotodioderne. Der er ligesom for den kapacitive sensor en akrylplade som koppen placeres på}
    \label{fig:optic_opstilling}
\end{figure}

Til at lave målinger der er sammenlignelige med de målinger der laves til den kapacitive sensor, skal den bedste afstand mellem IR LED og fotodioder bestemmes. Dette gøres ved at måle signalet når der ikke er en kop, når der er en kop med øl, når der er en bold som flyder i øllen i centrum af koppen og når der er en bold som flyder i øllen i kanten af koppen (over en af fotodioderne). Disse målinger udføres ved afstande mellem IR LED og hver af fotodioderne i intervallet 7mm til 28mm. Resultatet kan ses på \ref{fig:optic_test_afstand_raw}, hvor spændingen i forhold til referencen på 1.024V er plottet.
Målingen uden en kop, kan sammenlignes med det Cypress kalder baseline i CapSense. Denne måling trækkes fra de andre målinger. Herefter fås det der ses på figur \ref{fig:optic_test_afstand}. Derudover beregnes også den procentvise ændring ved at der er en bold i koppen. Dette kan ses på figur \ref{fig:optic_test_afstand_procent}

\begin{figure}[H]
    \centering
    \includegraphics[width=\textwidth]{HardwareDesign/CupSensor/graphics/OpticTest/beer_afstand_raw.PNG}
    \caption{Testresultater, der viser signalstyrken i forhold til referencen på 1.024V som funktionen af afstanden mellem LED og fotodiode}
    \label{fig:optic_test_afstand_raw}
\end{figure}

\begin{figure}[H]
    \centering
    \includegraphics[width=\textwidth]{HardwareDesign/CupSensor/graphics/OpticTest/beer_afstand.PNG}
    \caption{Testresultater, der viser signalstyrken som funktionen af afstanden mellem LED og fotodiode}
    \label{fig:optic_test_afstand}
\end{figure}

\begin{figure}[H]
    \centering
    \includegraphics[width=\textwidth]{HardwareDesign/CupSensor/graphics/OpticTest/beer_afstand_procent.PNG}
    \caption{Testresultater, der viser den procentvise ændring ved at tilføje en bold til koppen som funktionen af afstanden mellem LED og fotodiode}
    \label{fig:optic_test_afstand_procent}
\end{figure}

Der ses på figur \ref{fig:optic_test_afstand} at signalet bliver større jo tættere dioderne er på hinanden. Desuden ses det på \ref{fig:optic_test_afstand_raw} at 'baseline' signalet (signalet uden en kop) også stiger jo tættere dioderne er på hinanden. Derudover ses det på figur \ref{fig:optic_test_afstand_procent} at der sker en større procentvis ændring i signalet, når der er en bold i koppen, jo længere dioderne er fra hinanden. Sensorens hovedformål er at detektere placering af en kop, der skal derfor være et relativt stort og pålideligt signal når der er en kop. Derfor vælges det at arbejde videre med afstanden på ca 10mm, her er et relativt stort signal, sammenlignet med en peak-peak støjværdi på ca 40mV. Der er stadig en relativ stor procentvis ændring på 38\% til 110\%. Sammenlignet med 17\% for den kapacitive sensor.

Der testes nu den transiente respons ved placering af kop og ved tab af bold i kop. På figur \ref{fig:optic_place_and_drop_110ml} vises resultatet af at placere en kop på sensoren efterfulgt af at tabe en bold i koppen.

\begin{figure}[H]
    \centering
    \includegraphics[width=\textwidth]{HardwareDesign/CupSensor/graphics/OpticTest/placingCupAndDroppingBall(110ml-beer).png}
    \caption{Transient repsons for placering af kop (ca ved t=-3,5s)  og tab af bold (ca ved t=-0,3s)}
    \label{fig:optic_place_and_drop_110ml}
\end{figure}


\textbf{Diskussion af resultater}\\
Der ses på \ref{fig:optic_place_and_drop_110ml} at der kommer en spids, når der placeres en kop. Denne spids kommer også over der stationære niveau for når der er en bold i koppen. Derudover kommer der en meget stor spids når bolden rammer i koppen. Dette kan muligvis forklares ved at bolden kommer meget tæt på LED og fotodiode. Denne spids er meget stor, det er en procentvis ændring fra når der er en kop på ca $\frac{1.1\si{V}}{0.4\si{V}} = 275\%$, hvilket er meget større end den kapacitive sensor på ca. 22\%. Denne spids kan derfor bruges til at detektere at en bold rammer i. 

\subsubsection{Signal-støjforhold}
Som en del af valg af den bedste teknologi (næste afsnit) diskuteres der bl.a. signal-støjforhold for den kapacitive sensor og den optiske sensor. Disse diskussioner bygger ikke på det bedste grundlag, da målingerne blev udført før et ordentligt kendskab til signal-støjforhold, og der blev derfor ikke målt RMS værdien af støjen. Der laves dog stadig overvejelser omkring signal-støjforholdet, ud fra det man kan se på graferne. Hvis man sammenligner de to grafer som ses på \ref{fig:optic_place_and_drop_110ml} og øverst på figur \ref{fig:cap_test_place_and_drop_110} ses det at der er mere støj for den optiske sensor i forhold til hvor stor signal ændring der er (ændring mellem der hvor der ikke er en kop, og der hvor der er en kop). For optiske sensor er der en tyk 'streg' hvilket betyder at der meget støj, da signalet svinger meget op og ned, hvorimod der er en meget tynd 'streg' for den kapacitive sensor. Disse observationer kan derfor indikere at den kapactive sensor har et større signal-støjforhold. Dog er der desvære ikke målinger der efterviser dette.

\subsubsection{Valg af teknologi}
Begge teknologier virker til at være i stand til at detektere at at kop placeres på og muligvis også at en bold rammer i.
Der vælges nu en af de to teknologier. Den kapcitive sensor har et højere signal-støjforhold (SNR) end den optiske sensor. Hvis der udelukkende vælges på bagrund af SNR, vil det være bedst at vælge den kapacitive sensor. Men da det også er et 'should' krav at være i stand til at detektere at en bold rammer i koppen, medtages muligheden for dette også i valget af teknologi. Undersøgelserne viste at der med den kapcitive sensor ikke kommer den store ændring i signalet når en bold rammer i. Med den optiske sensor er der en større ændring, og en endnu større kortvarig ændring, det vil derfor være nemmere at detektere at en bold rammer i vha. den optiske sensor. Desuden er det ud fra undersøgelsen af den optiske sensor muligt at justere hvor stor ændring der skal være (på bekostning af SNR). Dette vil muligvis også være muligt med den kapacitive sensor, men det er svært at analysere sig frem til den bedste løsning, og omkostningsfuldt at teste mange forskellige løsninger (både tid og ressourcer). 

Sensoren skal kun detektere 3 forskellige tilstande: der er ikke en kop, der er en kop og der er en kop med en bold i. Dette kræver ikke nødvendigvis så stor en SNR. Hvis der derimod skulle måles hvor meget øl der er i en kop, vil det være vigtigt med en høj SNR. Med den optiske sensor er der større ændring i signalet når en bold rammer i.

En meget stor fordel for den kapacitive sensor er at den er meget billig, der skal kun benyttes en PSoC (som allerede skal bruges til andre dele af systemet) og en metal plade, dette kan nemt være som en del af en større printplade. Hvis der i en senere model ikke ønskes at bruge PSoC og dermed CapSense, kan det måske løses med en billigere løsning. Det kan måske være muligt at måle kondensatoren vha. et simpelt RC led, hvor stigetiden måles. Det kan derfor være meget billigt med en kapacitiv sensor. Til en optisk sensor skal der derimod benyttes en del flere komponenter som dermed gør systemet dyrere.
Det er svært at vælge en teknologi på grundlag af prisen, da der ikke er sat nogen specifikke krav til prisen af produktet. Derfor vælges den optiske sensor, da det gør det meget realistisk at være i stand til at detektere både en kop og at en bold rammer i.

\subsubsection{Diskussion} \label{sec:CupSensorTekUnderDiskussion}
Undervejs i teknologiundersøgelsen blev der observeret forskellige ting som er interesant til den videre udvikling. 
I testen hvor forholdet mellem signalstyrke og afstanden mellem LED og fotodiode blev undersøgt, blev det observeret at signalet varierer afhængig af hvor i kanten bolden er. Altså hvis bolden er lige over en fotodiode er signalet højt, men hvis den drejes 90 grader rundt om centrum af koppen, så den er midt i mellem de to fotodioder (men stadig i kanten af koppen), er signalet lavere. Ud fra dette kan det erfares at det vil være fordelagtigt at have flere fotodioder i en cirkel rundt om den infrarøde LED. Dette vil dog være spild at have alt for mange fotodioder, så det tænkes at 4 er tilstrækkelig.
Som en del af undersøgelsen af hvordan CapSense virker, blev deres signalbehandling beundret. Den måde CapSense detektere om der er en finger/kop/mm. vha. baselines virker som en smart signalbehandling. Det overvejes derfor at benyttes denne type signalbehandling til brug ved den optiske sensor.

\subsubsection{Konklusion} \label{sec:CupSensorTekUnderKonklusion}
I denne teknologiundersøgelse er der undersøgt specielt to forskellige teknologier, kapacitiv sensor og optisk sensor. Der blev valgt den optiske sensor, som er undersøgt nok til et 'proof of concept', der gør at der i den videre udvikling kan bevares en meget godt tiltro at teknologien vil virke. Men den skal stadig videreudvikles.

\iffalse
\subsection{Nærmere undersøgelse}
\subsubsection{Maksimal mulig strøm fra en fotodiode (uden AC coupling)} \label{sec:CupSensorCurrentTest}
{
Dette afsnit bygger på et design som det fra teknologiundersøgelsen. Dette er ikke det endelige design og dette afsnit er derfor ikke så brugbart længere. Det kan dermed springes over.

Til at bestemme om forskellige designs er mulige, er det nødvendigt at kende strømmen fra fotodioden i forskellige tilfælde.
Der benyttes fotodioden SFH 203 FA. I databladet for denne står der at photocurrent er $50 (\geq 30) \si{mA}$ men dette er for en bestem bølgelængde og lysintensitet. Det er ikke tydeligt om det er den maksimale strøm, derfor måles denne.
Det testes nu hvad den maksimal mulige strøm fotodioden kan levere når den direkte belyses af LED'en. Dette er ikke nødvendigvis det maksimale den kan levere, men det antages at fotodioden aldrig vil blive belyst med mere end dette.
LED'en tændes og strømmen måles ved hjælp af kredsløbet som ses på figur \ref{fig:PhotodiodeTestDiagram}. Strømmen måles ved at måle spændingen over formodstanden. Formodstanden er en parallel kobling af en 68Ohm modstand og en 51Ohm modstand som svare til 29Ohm. Spændingen over modstandene måles til $3.06\si{V}$ og dermed er den samlede strøm gennem dem og dermed LED'en ca $106\si{mA}$. Denne strøm måles da det er ca denne strøm der regnes med at blive brugt. Strømmen fra fotodioden måles vha. TIA'en som ses på figur \ref{fig:PhotodiodeTestDiagram}. Her måles spændingen $4.65V$. Dette er det maksimale output af TIA'en og forstærkningen kan ikke sænkes. Ud fra dette resultat kan det konkluderes at strømmen fra photdioden mindst er $230\si{\mu A}$

\begin{figure}[H]
    \centering
    \includegraphics[width=\textwidth]{HardwareDesign/CupSensor/graphics/DiodeCurrentTestDiagram.PNG}
    \caption{Diagram for test kredsløb}
    \label{fig:PhotodiodeTestDiagram}
\end{figure}

Opstillingen ændres nu til det som ses på figur \ref{fig:CurrentTestUpDown}. TIA'ens feedback modstand ændres til $40\si{k\Omega}$ Der benyttes et spejl som bevæges op og ned over photodiode/LED og den maksimale værdi fra TIA'en bestemmes til $3.89\si{V}$. Den maksimale strøm med denne opstilling er derfor $97\si{\mu A}$.

Der benyttes nu samme opstilling men der benyttes en plastikkop med en bordtennisbold (her måles en større strøm end en kop med væske) i stedet for et spejl. TIA'ens feedback modstand ændres til $120\si{k\Omega}$. Koppen bevæges op og ned over photodiode/LED og den maksimale værdi fra TIA'en bestemmes til $2.98\si{V}$. Den maksimale strøm med denne opstilling er derfor $25\si{\mu A}$.

Der benyttes nu opstillingen som kan ses på figur \ref{fig:CurrentTestBeer}. Der er 110ml øl i koppen. TIA'ens feedback modstand ændres til $250\si{k\Omega}$. En bordtennis bold kastes ned i koppen den maksimale værdi fra TIA'en bestemmes til $1.86\si{V}$. Den maksimale strøm med denne opstilling er derfor $7.4\si{\mu A}$.

Der benyttes nu samme opstilling men akrylpladen flyttes helt ned til LED/fotodiode (før var den ca $4\si{mm}$ fra LED/fotodiode) \ref{fig:CurrentTestBeer}. TIA'ens feedback modstand er stadig $250\si{k\Omega}$. En bordtennis bold kastes ned i koppen den maksimale værdi fra TIA'en bestemmes til $1.82\si{V}$ (hvilket er stort set det samme som ved den tidligere test. Den maksimale strøm med denne opstilling er derfor $7.3\si{\mu A}$. \\
Det noteres desuden at spændingen fra TIA'en er $158\si{mV}$ (strømmen er dermed $0.6\si{\mu A}$ når der ikke er nogen kop, i modsætningen til når pladen er hævet ca $4\si{mm}$ hvor spændingen fra TIA'en er $675\si{mV}$ (strømmen er dermed $2.7\si{\mu A}$. Dette er ikke formålet med denne test, men det betyder at akrylpladen skal placeres så tæt på LED og fotodiode så muligt, da signalændringen bliver større når akrylpladen er tæt på LED og fotodiode. 

Der benyttes nu samme opstilling men der benyttes en hånd i stedet for en kop med øl. TIA'ens feedback modstand ændres til $120\si{k\Omega}$. Håndes bevæges op og ned over photodiode/LED og den maksimale værdi fra TIA'en bestemmes til $2.36\si{V}$. Den maksimale strøm med denne opstilling er derfor $20\si{\mu A}$.

Resultaterne vises på tabel \ref{tab:CupSensorCurrentTest}

\begin{table}[H]
\begin{tabular}{|L{0.25\textwidth}|L{0.25\textwidth}|L{0.25\textwidth}|L{0.25\textwidth}|}
\hline
\textbf{Måleopstilling} & \textbf{TIA feedback modstand} & \textbf{Maximal TIA udgangsspænding} & \textbf{Maksimal Strøm fra fotodiode} \\ \hline
\textbf{LED lyser ind i fotodiode} & $20\si{k\Omega}$ & $4.65\si{V}$ & $230\si{\mu A}$ \\ \hline
\textbf{Spejl} & $40\si{k\Omega}$ & $3.89\si{V}$ & $97\si{\mu A}$ \\ \hline
\textbf{Plastikkop med bold} & $120\si{k\Omega}$ & $2.98\si{V}$ & $25\si{\mu A}$ \\ \hline
\textbf{Plastikkop med bold, øl og akrylplade} & $250\si{k\Omega}$ & $1.86\si{V}$ & $7.4\si{\mu A}$ \\ \hline
\textbf{Plastikkop med bold, øl og akrylplade helt tæt på LED og fotodiode} & $250\si{k\Omega}$ & $1.82\si{V}$ & $7.3\si{\mu A}$ \\ \hline
\textbf{Hånd og akrylplade helt tæt på LED og fotodiode} & $120\si{k\Omega}$ & $2.36\si{V}$ & $20\si{\mu A}$ \\ \hline
\end{tabular}
\caption{Testresultater for forskellige opstillinger}
\label{tab:CupSensorCurrentTest}
\end{table}
}
\fi


\subsection{Detektering af lys}
Til at detektere lys bruges der som tidligere beskrevet en fotodiode som skal detektere lys fra en IR LED. 
Det er valgt at benytte fotodioden SFH203FA som er mest sensitiv overfor infrarødt lys (900 nm)\autocite[2]{SFH203FA} og passer meget godt til lyset som kommer fra LED'en SFH485 (880nm)\autocite[3]{SFH485}. Det er valgt at benytte infrarødt lys, da der i normale omgivelser vil være større forstyrelser i det synlige område, fx fra kunstig belysning. De valgte komponenter er valgt da de passer godt til hinanden samt at det er dem, som der er tilrådighed. 
Da der altid vil være forstyrelser fra andre kilder er det valgt at LED'en skal blinke med en given frekvens og der kun måles signaler fra fotodioden ved denne frekvens. Til dette vil der benyttes en mixer, som vil blive forklaret senere. 

\subsubsection{Fotodiode beskrivelse}
Først beskrives virkemåden af en fotodiode. En fotodiode fungerer som det ses på figur \ref{fig:photodiode_operation}. En fotodiode har den karakteristiske spænding-strøm forhold som en normal diode. Dette forhold ændres når fotodioden belyses. Her rykkes hele grafen ned, alt efter lysintensiteten. Der begynder altså at løbe en strøm i spærreretningen. Hvis spændingen over dioden holdes konstant vil strømmen der løber i spæreretningen afhænge af lysintensiteten. 
\begin{figure}[H]
    \centering
    \includegraphics[width=\textwidth]{HardwareDesign/CupSensor/graphics/Photodiode_operation.png}
    \caption{Funktionalitet for en fotodiode}
    \label{fig:photodiode_operation}
\end{figure}

Til at benytte en fotodiode, ønskes en simpel model for en fotodiode. En model kunne være den som ses på figur \ref{fig:photodiodeModel}.

\begin{figure}[H]
    \centering
    \includegraphics[width=\textwidth]{HardwareDesign/CupSensor/graphics/photodiodeModel.png}
    \caption{Model for fotodiode}
    \label{fig:photodiodeModel}
\end{figure}

Det har ikke være muligt at finde tilstrækkelig oplysninger i databladet for den benyttede fotodiode til at benytte denne model. Fx er $R_D$ ikke angivet nogen steder. Det antages derfor at R\_D er så stor at den ikke har nogen betydning. Der benyttes i stedet en mere simpel model bestående kun af en strømkilde og en kondensator. som det ses på figur \ref{fig:photodiodeModelSimple}.

\begin{figure}[H]
    \centering
    \includegraphics[width=0.9\textwidth,trim={4.1in 2.1in 4.75in 4.7in},clip, page=1]{HardwareDesign/CupSensor/graphics/Superposition.pdf}
    \caption{Anvendt model for fotodiode}
    \label{fig:photodiodeModelSimple}
\end{figure}

\subsubsection{AC kobling}
Strømmen ønskes at laves om til en spænding. Men kun AC komponent af signalet, da LED'en blinker vil nyttesignalet være et AC signal. Der benyttes standard kredsløbet fra øvelse 6 i MSE. Dette kan ses på figur \ref{fig:photodiodeCircuit}. DC komponenten af signalet filtreres væk vha. RC ledet. Den resterende AC strøm laves om til en spænding vha. en TIA. På \ref{fig:photodiodeCircuit} ses at der er en modstand R, som bl.a. sørger for at der hele tiden er påtrykt en spænding i spæreretningen. Dette er en af ændringerne i forhold til kredsløbet fra teknologiundersøgelsen. Spændingen i spæreretningen sørger for at formindske kondensatoren $C_D$ i \ref{fig:photodiodeModelSimple}. Det at kondensatoren formindskes kan ses ud fra databladet for SFH485 (se \ref{fig:C_D_graph}). Det ses at $C_0$ (i vores model $C_D$) bliver mindre jo større $V_R$ er. $V_R$ er 'reverse voltage' og altså spændingen i spæreretningen. Ved at tilføje modstanden R, sørges der for at $V_R$ altid er positiv. Dermed er $C_0 = C_D \approx 3.5\si{pF}$ ved $V_R=5\si{V}$. Det ønskes at sænke kapacitansen for $C_D$ da dette vil øge båndbredden for dioden da kondensatoren ikke skal påføres så stor en ladning. Derudover påvirker en kondensator på indgangen af en operationforstærker også stabiliteten af denne operationsforstærker.

\begin{figure}[H]
    \centering
    \includegraphics[width=0.9\textwidth,trim={0.6in 5.1in 6.0in 0.55in},clip, page=1]{HardwareDesign/CupSensor/graphics/Superposition.pdf}
    \caption{Diagram for fotodiode kredsløb}
    \label{fig:photodiodeCircuit}
\end{figure}

Kredsløbet fungerer på sådan en måde at det offset der vil være fra fotodioden, grundet mørkestrømmen (dark current) og lys fra omgivelserne, bliver filtreret fra vha. AC kobling kondensatoren $C_{coupling}$. Dette vil nu forklares vha. superposition. Men først laves en model for kredsløbet, i denne model indgår både modellen for dioden og en model for TIA benyttes, som er en strømstyret spændingskilde. Denne model kan ses på figur \ref{fig:photodiodeCircuitModel}. 
Da der er negativ feedback på operationsforstærkeren vil det svare til at den ikke-inverterende indgang har samme spænding som den inverterende indgang. Dette er også medtaget på modellen. Derudover er forstærkningen for den strømstyret spændingsforsyning $-R_{feedback}$. Det ses på diagrammet at strømmen som styre spændingen er strømmen igennem $C_{coupling}$. Denne strøm noteres $I_{coupling}$ og spændingen på den strømstyret spændingkilde er derfor$-R_{feedback} I_{coupling}$.

\begin{figure}[H]
    \centering
    \includegraphics[width=0.9\textwidth,trim={0.6in 2.4in 7in 3.5in},clip, page=1]{HardwareDesign/CupSensor/graphics/Superposition.pdf}
    \caption{Model for fotodiode kredsløb}
    \label{fig:photodiodeCircuitModel}
\end{figure}

Der benyttes nu superposition hvor alle andre kilde end Vref slukkes (udover selvfølgelig den afhængige kilde). Til dette er der lavet et diagram som beskriver situationen, som kan ses på figur \ref{fig:photodiodeSPVref}. Der ses at $V_{ref}$ er forbundet til $C_{coupling}$, og da $V_{ref}$ er en DC spænding og en kondensator fungere som et åbent kredsløb ved DC, løber der derfor ikke nogen strøm ($I_{coupling\_Vref}=0$) og dermed er $V_{out\_Vref}=V_{ref}$ Der bruges notationen hvor Vref skrives i subscripten for at indikere at det er outputtet når kun $Vref$ er tændt.

\begin{figure}[H]
    \centering
    \includegraphics[width=0.9\textwidth,trim={6.3in 1.3in 1in 5in},clip, page=1]{HardwareDesign/CupSensor/graphics/Superposition.pdf}
    \caption{Fotodiode kredsløb når alle andre kilder end $Vref$ slukkes}
    \label{fig:photodiodeSPVref}
\end{figure}

Herefter benyttes superposition hvor kun VDD er tændt. Til dette er der lavet et digram som beskriver situationen, som kan ses på figur \ref{fig:photodiodeSPVDD}. Der ses igen at VDD er forbundet til $C_{coupling}$ og ligesom før, løber der ingen strøm igennem den ($I_{coupling\_VDD}=0$). Dermed er $V_{out\_VDD}=0$

\begin{figure}[H]
    \centering
    \includegraphics[width=0.9\textwidth,trim={6.3in 3.5in 0.8in 2.7in},clip, page=1]{HardwareDesign/CupSensor/graphics/Superposition.pdf}
    \caption{Fotodiode kredsløb når alle andre kilder end $VDD$ slukkes}
    \label{fig:photodiodeSPVDD}
\end{figure}

Til sidst undersøges situationen hvor kun $I_D$ er tændt. Til dette er der lavet et digram som beskriver situationen, som kan ses på figur \ref{fig:photodiodeSPI_D}. 

\begin{figure}[H]
    \centering
    \includegraphics[width=0.9\textwidth,trim={6.3in 5.8in 0.8in 0.6in},clip, page=1]{HardwareDesign/CupSensor/graphics/Superposition.pdf}
    \caption{Fotodiode kredsløb når alle andre kilder end $VDD$ slukkes}
    \label{fig:photodiodeSPI_D}
\end{figure}

Der ses at der er en parallel kobling af de fire komponenter $I_D$, $R$, $C_D$ og $C_{coupling}$. Derfor kan strømmen $I_{coupling\_I\_D}$ bestemmes vha. strømdeling. 
$$
I_{coupling\_I\_D} = - I_D \frac{Y_{coupling}}{Y_R + Y_D + Y_{coupling}}
$$
hvor
$$
Y_R = \frac{1}{R} \textrm{,} \quad Y_D = C_D \dot s \quad \textrm{og} \quad Y_{coupling} = C_{coupling} s
$$
Dermed er strømmen
\begin{align}
I_{coupling\_I\_D} &= -I_D \frac{C_{coupling} s}{\frac{1}{R} + C_D s + C_{coupling} s}\\
&= - I_D \frac{R C_{coupling} s}{1 + R\left(C_D + C_{coupling} \right)s}\\
&=  - I_D \frac{\frac{R C_{coupling}}{R\left(C_D + C_{coupling} \right)} s}{\frac{1}{R\left(C_D + C_{coupling} \right)} + s}\\
&= - I_D \frac{C_{coupling}}{C_D + C_{coupling}} \frac{s}{\frac{1}{R\left(C_D + C_{coupling} \right)} + s}
\end{align}

Dermed er udgangen

\begin{align}
V_{out\_I\_D} &= -R_{feedback}I_{coupling\_I\_D}\\
&= I_D \frac{C_{coupling}R_{feedback}}{C_D + C_{coupling}} \frac{s}{\frac{1}{R\left(C_D + C_{coupling} \right)} + s}
\end{align}
Hvis $C_D << C_{coupling}$ kan det aproksimeres til
$$V_{out\_I\_D} = I_D R_{feedback} \frac{s}{\frac{1}{R C_{coupling}} + s}$$

Det ses at dette er et 1. ordens højpasfilter med cutoff frekvensen 
\begin{align}
\omega_c &= \frac{1}{RC_{coupling} }\\
&= \frac{1}{10\si{k\Omega} \cdot 0.1\si{\mu F}}\\
&= 1000 \si{\frac{rad}{s}}\\
f_c &= 159 \si{Hz}
\end{align}

Ud fra superposition kan det konkluderes at så læge $VDD$ og $V_{ref}$ er konstante og $C_D << C_{coupling}$ vil der kun detekteres AC komponenten fra $I_D$ med et offset på $V_ref$. Eller udtryk som nedenfor.
$$V_{out} = I_D R_{feedback} \frac{s}{\frac{1}{R C_{coupling}} + s} + V_{ref}$$

Dette er tildels hvad der ønskes, da LED'en blinker skal der kun detekteres AC komponenten fra fotodioden. Men det ønskes kun at detektere den del af signalet som har samme frekvens som LED'en blinker med. Til dette skal der laves et båndpas filter.

\subsubsection{Mixer}
Til at lave et båndpas filter er det valgt at benytte en mixer hvor lokaloscillatoren (LO) er den samme oscillator som styrer LED'en, dvs. samme frekvens og fase. En del af det endelige hardware design på PSoC'en kan ses på \ref{fig:final_PSoC_design_input_part}. Der er på denne figur kun medtaget den del som håndtere input, og altså ikke styringen af LED'en. Dog skal det siges at udgangen fra LED\_clock styrer om LED'en er tændt eller slukket.

\begin{figure}[H]
    \centering
    \includegraphics[width=1\textwidth]{HardwareDesign/CupSensor/graphics/Final_PSoC_Design_Input_part.PNG}
    \caption{Endelig hardware design på PSoC. (Kun input delen er vist). Udgangen fra AC koblingen forbindes til photodiodesIn}
    \label{fig:final_PSoC_design_input_part}
\end{figure}

En ting der kan ses på diagrammet er, at der ikke er nogen mixer. Dette er fordi ADC'en har en indbygget funktion som kan få den til at have funktionaliteten af en mixer \autoref[3]{ADC-DelSig-datasheet}. Det er valgt da det formindsker antallet af nødvendige komponenter. ADC'ens mixer funktionalitet aktiveres ved at vælge "Enable modulator input" i konfigurationen af ADC'en. Når denne er valgt, har ADC'en inputtet 'mi'. Når dette input er højt, inverteres polariteten modulatoren i delta sigma ADC'en. På denne måde fungere ADC'en som en mixer, udover at en normal mixer vil invertere inputtet når LO er lav, derfor er der tilføjet en inverter mellem LED\_clock og 'mi' inputtet.

Når der benyttes samme clock til at styre LED og til LO, er udgangen fra mixeren et DC niveau, når der ikke er støjsignaler tilstede. Men for at dette DC niveau ikke er nul. Er det vigtigt at signalet til mixeren og LO har samme fase. Det undersøges nu derfor om dette er tilfældet.

Det undersøges om lysets forsinkelse vil have en betydelig indflydelse på fasen af signalet fra fotodioden. Tidsforsinkelsen for lyset vil være 
$$t_d = \frac{l}{v_l}$$
hvor $t_d$ er tidsforsinkelse, $l$ er afstanden lyset bevæger sig og $v_l$ er hastigheden af lyset. Hvis det antages at hele den afstand, som lyset bevæger sig foregår i øl, som antages at være vand, er lysets hastighed $v_p = \frac{c}{n_{vand}}$, hvor $c$ er lysets hastighed i vakuum og $n_{vand}$ er brydningsindekset for vand.
Dermed er tidsforsinkelsen
$$t_d = n_{vand}\frac{l}{c}$$
fasen er dermed
$$\phi_{lys} = -2 \pi f_{LED\_clock} n_{vand}\frac{l}{c}$$, hvor $\phi_{lys}$ er fasen grundet lysets forsinkelse og $f_{LED\_clock}$ er frekvensen LED'en blinker med. Ved indsættelse af værdierne $n_{vand}=1.33$ \autocite{brydningsindex}, $f_{LED\_clock} = 10kHz$ og $l=1m$, som er meget længere end hvad det reelt vil være, fås fasen til 
$$\phi_{lys} = -0.032 \si{^{\circ}}$$

Det undersøges om AC koblingen vil have en betydelig indflydelse på fasen. I et tideligere afsnit blev nedenstående udtryk bestemt
$$V_{out\_I\_D} = I_D R_{feedback} \frac{s}{\frac{1}{R C_{coupling}} + s}$$
Ved indsætningen af værdier for $R=10k\Omega$, $C_{coupling} = 0.1 \si{\mu F}$ og $s=j2\pi f_{LED\_clock} = j2\pi 10 \si{kHz}$ fås det at fasen er 
$$\phi_{ac\_kobling} = 0.912\si{^{\circ}}$$

Det kan konkluderes at der stort set ikke er nogen faseforskel. Den lille faseforskel der er, vil forårsage at DC niveauet vil være en lille smule mindre, men den vil være tæt på den maksimale mulige. 

Formålet med at bruge en mixer er at lave et båndpasfilter. For dette er tilfældet skal der på udgangen af mixeren være et lavpasfilter. Dette er også inkluderet i ADC'en som et sinc 4 filter \autocite[35]{ADC_DelSig_datasheet}. Filteret virker som et lavpasfilter som vil have vil have uendelig stor dæmpning ved samplefrekvensen og alle dens harmoniske overtoner. Samplefrekvensen indstilles til 2500 samples/s. I kombination med mixeren vil det virke som et båndpas filter.  

\newpage
\subsection{Håndtering af flere sensorer}

Der overvejes forskellige metoder til at håndtere flere sensorer.
\subsubsection{Flere kanaler}
Det overvejes at have én kanal til hver sensor. Et udkast til sådan en løsning kan ses på \ref{fig:multiple_channels}. Det vil sige at LED'erne på de forskellige sensorer (LED1 og LED2) blinker med forskellige frekvenser (Clock\_1 og Clock\_2). På denne måde kan man indstille kanalen på mixeren til den kanal der passer til den sensor der ønskes at blive læst. Dette gøres ved at indstille den lokale oscillator til mixeren (Clock\_3). Hvis det ønskes at læse fra sensoren bestående af LED1 og D1 indstilles Clock\_3 til at have samme frekvens som Clock\_1 (i eksemplet 11kHz). Denne løsning vil betyde at alle fotodioderne fra alle sensorerne kan forbindes til det samme ben på PSoC'en, hvilket er fordelagtigt.

\begin{figure}[H]
    \centering
    \includegraphics[width=1\textwidth]{HardwareDesign/CupSensor/graphics/Flere_kanaler.PNG}
    \caption{Udkast til hvordan kredsløbet kan se ud hvis der benyttes flere kanaler. Der er kun medtaget to sensorer(LED1 og D1 og den anden sensor er LED2 og D2). Der er for overskuelighedens skyld kun en fotodiode per sensor}
    \label{fig:multiple_channels}
\end{figure}

 En ulempe ved denne metode er, at TIA'en skal have en relativ lav forstærkning, for at den ikke går i mætning i de tilfælde hvor LED'en på alle sensorer er tændt på samme tid. Udgangen på mixeren vil være relativt lav da det jo kun er er fra en sensor. Der vil derfor kræves yderligere forstærkning på udgangen af mixeren. Hvis TIA'en derimod er indstillet til at forstærke signalet fra kun én sensor (som i de andre metoder beskrevet), kan denne forstærkning være ca 6 gange større.
 
 En fordel ved denne metode er at klok-signalet til hver sensor ikke nødvendigvis behøver at være på PSoC'en, man kunne have et lokalt kloksignal ved hver sensor for at minere brugen af ben på PSoC'en.

\subsubsection{Multiplexing}
En anden metode til at håndtere flere sensorer, er kun at sende kloksignalet til LED'en på én sensor af gangen og samtidig kun måle signalet fra fotodioderne på én sensor af gangen. Et udkast til sådan en løsning kan ses på \ref{fig:multiplexing}. Til dette skal der bruges 6 digitale udgange og 6 analoge indgange og tilsammen 12 ben på PSoC'en. Dette er en af ulemperne med denne løsning. Der ses på \ref{fig:multiplexing} at der benyttes en multiplexer og en de-multilexer til at sende kloksignalet til en sensor og kun at modtage fra en sensor. Multiplexeren og de-multiplexeren skal således styres sammen, så der læses signalet fra den samme sensor som der sendes et signal til.

\begin{figure}[H]
    \centering
    \includegraphics[width=1\textwidth]{HardwareDesign/CupSensor/graphics/Multiplexing.PNG}
    \caption{Udkast til hvordan kredsløbet kan se ud hvis der benyttes multiplexing. Der er kun medtaget to sensorer(LED1 og D1 og den anden sensor er LED2 og D2). Der er for overskuelighedens skyld kun en fotodiode per sensor}
    \label{fig:multiplexing}
\end{figure}

\subsubsection{Multiplexing med kun en indgang}
Der overvejes en anden variation af multiplexing. Et udkast til sådan en løsning kan ses på \ref{fig:multiplexing_en_indgang}. Der sendes stadig kloksignal til en sensor af gangen, men alle sensorers fotodioder forbindes til den samme analoge indgang på PSoC'en. Her vil det blive antaget at lyset der bliver sendt ud fra en sensor ikke påvirker fotodioderne på de andre sensorer. Hvis denne antagelse er gyldig kan alle sensorer måles men med færre ben på PSoC'en end den tideligere beskrevet multiplexing metode. En ulempe til denne metode er at antagelsen om at de andre fotodioder ikke bliver påvirket, muligvis ikke er gyldig.

\begin{figure}[H]
    \centering
    \includegraphics[width=1\textwidth]{HardwareDesign/CupSensor/graphics/Multiplexing_en_indgang.PNG}
    \caption{Udkast til hvordan kredsløbet kan se ud hvis der benyttes multiplexing med kun en indgang. Der er kun medtaget to sensorer(LED1 og D1 og den anden sensor er LED2 og D2). Der er for overskuelighedens skyld kun en fotodiode per sensor}
    \label{fig:multiplexing_en_indgang}
\end{figure}


\subsubsection{Udvidelse til Multiplexing med kun en indgang}
Der overvejes en udvidelse af multiplexing med kun en indgang. I denne udgave vil det kloksignal der sendes til LED'en på en sensor også styre om fotodiodernes strøm skal sendes videre til TIA'en. Så der kun modtages et signal fra den sensor der læses. En ulempe for denne metode er at der skal benyttes flere komponenter end tidligere metode, og systemet bliver mere kompliceret. En fordel er at det ikke er nødvendigt at antage at de andre fotodioder ikke bliver påvirket, da de ikke vil sende signalet videre til TIA'en, og det bliver derfor mere pålidligt.

\subsubsection{Valg af metode}
Det vil umiddelbart være bedst at benytte den metode hvor der er én kanal til hver sensor, da det fx. kunne være muligt at have en oscillator til hver sensor ekstern fra PSoC'en, så der kun skal bruges ét ben på PSoC'en. Der er med denne metode problemet at forstærkningen vil være lille og signalet på udgangen af mixeren derfor vil være relativt lavt (dette kan dog forstærkes op). Derudover skal der benyttes flere forskellige kloksignaler på PSoC'en hvilket vil benytte flere recourcer på PSoC'en. Denne metode fravælges derfor. 

Den simple metode med multiplexing (6 udgange og 6 indgange) vil være relativ nem at implementere (i forhold til nogle af de andre metoder) og nok også rimelig pålidelig. Dette er en meget god grund til at vælge denne metode, men den bruger mange ben. Selvom det ikke er noget direkte krav hvor mange ben der må bruges, så er det overordnet i projektet et ønske (ikke beskrevet nogen steder) at benytte så få ben som muligt, da det vil muliggøre at benytte en mindre microcontroller/PSoC i et endeligt produkt (De PSoC kits der bruges i dette projekt, tænkes at blive erstattet af en billigere og mindre microcontroller/PSoC). Derfor fravælges denne metode også.

Til sidst er der de to metoder med multiplexing hvor der kun benyttes en indgang. Disse metoder benytte færre ben end den simple multiplexing metode. Der bruges 7 ben frem for 12. Samme antal ben som der vil bruges med metoden med flere kanaler. Derfor vælges en af disse metoder (en af multiplexing metoderne med kun en indgang). Forskellen mellem de to metoder er hovedsageligt at den ene metode bygger på antagelsen om at lys udsendt fra én sensor, ikke vil påvirke fotodioderne på de andre sensorer. Derfor vælges det at arbejde videre med den metode hvor det antages at denne antagelse er gyldig, og hvis det viser sig at den ikke er, kan det overvejes at skifte til den anden metode.
\newpage
\subsection{Signal conditioning}

\subsubsection{Maksimal mulig signalstyrke}
{
For at bestemme TIA'ens forstærkning måles signalet fra TIA'en ved forskellige testscenarier. Signalet måles med den højeste feedbackmodstand på TIA'en uden at forstærkeren går i mætning. Fx går forstærkeren i mætning på figur \ref{fig:TIA_saturation}. På figuren ses også cursorere der hvor der måles værdier. Der måles den positive peak-værdi i slutningen af målingen af den givne sensor. og der måles den negative peak-værdi i slutningen af målingen. Begge er i fohold til TIA'ens referencespænding på ca. 2.3V. Måleobjekterne flyttes rundt og rystes evt. for at frembringe den størst mulige værdi.

IKKE FÆRDIG
}

DENNE DEL PASSER IKKE LÆNGERE TIL RESTEN AF DESIGN DOKUMENTET IGNORER DETTE AFSNIT TIL REVIEW
En del af signal condtioning er at hele ADC'ens spændingsområde benyttes. I afsnit \ref{sec:CupSensorErfaringer} blev det konkluderet at den maksimale strøm fra en fotodiode i normale omstændigheder er $25\si{\mu A}$  og for fire fotodioder $100\si{\mu A}$. Dette skal forstærkes op til den maksimale udgang på TIA'en (ca $4.6\si{V}$). Det vil sige at feedbackmodstanden på TIA'en skal være $\frac{4.6\si{V}}{100\si{\mu A}} = 46\si{k\Omega}$. Dermed vælges den nærmeste mulige værdi som er under $46\si{k\Omega}$, dermed $40\si{k\Omega}$. Dette vil betyde at den højeste spænding på udgangen af TIA'en vil være $40\si{k\Omega} \times 100\si{\mu A} = 4\si{V}$.

Et 'sideresultat' fra afsnit \textbf{\ref{sec:CupSensorCurrentTest}} er at der med en akrylplade helt tæt på LED og fotodiode er en 'offset' strøm på ca. $0.6\si{mu A}$, når der er fire fotodioder vil det være samlet $2.4\si{mu A}$. Dette vil forårsage en udgangsspænding på $96\si{mV}$. Dette 'offset' kan muligvis fjernes ved at ændre på ADC'ens range eller vha. en operationsforstærker. 'Offset'et anses for at være for lille til at dette er nødvendigt.

Dvs. at udgangen på TIA'en vil være i intervallet $[96\si{mV};4\si{V}]$. Hvis ADC'ens input range sættes til "Vssa to Vdda" som er ca. $0\si{V}$ til $4.7\si{V}$ når PSoC'en forsynes med USB. Dette vil betyde at ca. $83\%$ af ADC'en input range benyttes. Dette er selvfølgelig ikke helt optimalt, men det er tilfredsstillende.




\end{document}