\documentclass[Accepttestspecifikation/Accepttest_Main.tex]{subfiles}

\begin{document}
\subsection{Accepttest - Ikke funktionelle krav} \label{sec:UC1}


\begin{longtable}{|L{0.08\textwidth}|L{0.3\textwidth}|L{0.3\textwidth}|L{0.2\textwidth}|L{0.2\textwidth}|}
\hline
\textbf{Krav ID} & \textbf{Krav} & \textbf{Test} & \textbf{Forventet resultat} & \textbf{Faktisk resultat} \\ \hline
K\ref{kravspec:req:table-dimensions} & Beer Pong bordet skal være 240cm / 60cm / 70cm +/- 1cm  & Højden, bredden og længden af bordet måles med et målebånd med en tolerance på 1mm & 240cm / 60cm / 70cm & Intet bord at måle\\ \hline
K\ref{kravspec:req:ball-size} & Systemet skal kunne håndtere en \textbf{Ball} med en diameter på 40mm +/- 1mm & Dette testes som en del af testene af UC1, UC2 og UC4, da der benyttes en 40mm \textbf{Ball} & Systemet er ikke fuldt operationelt, men alle handlinger hvor som involvere en bold fungerer.  \\ \hline
K\ref{kravspec:req:display-size} & Display'ets billede skal være 24 tommer +/- 0.5 tommer diagonalt med et format på 16:9 med en opløsning på 1920x1080& Det aflæses i databladet for skærmen & 24 tommer tommer, med et format på 16:9 og en opløsning på 1920x1080 & Intet display\\ \hline
K\ref{kravspec:req:cup-size} & Systemet skal kunne håndtere \textbf{Cups} med en øvre diameter på $94\si{mm} \pm 3\si{mm}$ og en nedre diameter på $65\si{mm} \pm 3\si{mm}$ og en højde på $120mm \pm{5mm}$ & Dette testes som en del af testene af UC1, UC2 og UC3. Der benyttes af praktiske oversager kun en type \textbf{Cup}. Dette bruges derudover til at teste en række ikke funktionelle krav &  Systemet virker & Systemet er ikke fuldt operationelt, men alle handlinger hvor som involvere en kop fungerer.\\ \hline
K\ref{kravspec:req:coin-size} & Systemet skal kun acceptere en dansk femkrone\autocite{fiveKrCoin} som betaling. Diameter på 28.5 mm +/- 0.5mm og tykkelse på 2.00 +/- 0.1 mm & Dette testes som en del af testen for UC1 & Systemet acceptere en dansk 5kr som betaling & Der accepteres en 5 krone\\ \hline
K\ref{kravspec:req:coin-max-size}& Systemet skal frigive \textbf{Coins} med en diameter på 28.0 mm og under og tykkelse på 2.35 mm og under & Dette testes som en del af UC1. Til dette benyttes en coin med diameteren 19.6 og tykkelse på 3mm. Dette er ikke optimalt, da der ikke benyttes den rigtige størrelse mønt. & Systemet frigiver \textbf{Coin} & Ikke testet\\  \hline
K\ref{kravspec:req:light-D-inner}& $D_{inner}$ skal være $55\si{mm} \pm 1\si{mm}$ & Dimensionen måles med en skydelære med en opløsning på 1/2 mm & $55\si{mm}$ & $D_{inner}$ eksistere ikke. \\ \hline
K\ref{kravspec:req:light-D-center}& $D_{center}$ skal være $65\si{mm} \pm 1\si{mm}$  & Dimensionen måles med en skydelære med en opløsning på 1/2 mm & $65\si{mm}$ & Der er kun en LED\\ \hline
K\ref{kravspec:req:light-D-led} & $D_{led}$ skal være $5\si{mm} \pm 1\si{mm}$ & Dimensionen måles med en skydelære med en opløsning på 1/2 mm & $5\si{mm}$ & $4.92mm$ \\ \hline
K\ref{kravspec:req:light-D-outer}& $D_{outer}$ skal være $75\si{mm} \pm 1\si{mm}$ & Dimensionen måles med en skydelære med en opløsning på 1/2 mm & $75\si{mm}$ & eksistere ikke \\ \hline
K\ref{kravspec:req:light-led-angle}& Vinklen mellem to LED'er i \textit{\textbf{Cup light}} skal være $70^{\circ} \pm 2^{\circ}$ & Vinklen mellem hver led måles med en vinkelmåler med en opløsing på $1^{\circ}$ & $70^{\circ}$, $70^{\circ}$, $70^{\circ}$, $70^{\circ}$, $70^{\circ}$, $70^{\circ}$ &  Der er kun en LED \\ \hline
K\ref{kravspec:req:cup-holder-distance} & Afstanden mellem centrum for to \textit{\textbf{Cupholders}} skal være $100\si{mm} \pm 2\si{mm}$ & Dimensionen måles med en skydelære med en opløsning på 1/2 mm. Der måles mellem alle \textit{\textbf{Cupholders}} som er ved siden af hinanden & $100\si{mm}$, $100\si{mm}$, $100\si{mm}$, $100\si{mm}$, $100\si{mm}$, $100\si{mm}$, $100\si{mm}$, $100\si{mm}$, $100\si{mm}$ & $100.7\si{mm}$, $99.8\si{mm}$, $99.9\si{mm}$, $101.1\si{mm}$, $99.0\si{mm}$, $99.6\si{mm}$, $100.0\si{mm}$, $100.6\si{mm}$, $100.9\si{mm}$\\ \hline
K\ref{kravspec:req:beer-amount}& Systemet skal fungere med $110\si{ml} \pm 20\si{ml}$ Ceres Top øl i hver \textbf{Cup} & UC1, UC2 og UC3 testes med $90\si{ml}$ og $130\si{ml}$. Dette sammenlignes med resultaterne for testen der blev udført med $110\si{ml}$. & Ingen forskelle & Med 90ml: detekterede alle bolde ramte i. Den sidste kop skulle fjernes to gange, før der skiftes til at vise hvem der vandt. . 
Med 130ml: Samme som med 110ml\\ \hline
K\ref{kravspec:req:ball-color} & Systemet skal fungere med en hvid \textbf{Ball} & Til alle test der involvere en \textbf{Ball}, benyttes en hvid \textbf{Ball}& Systemet fungere & \\ \hline
K\ref{kravspec:req:ball-drop-height} & Systemet skal detektere en \textbf{Ball} der falder fra en højde på $30\si{cm} \pm 1\si{cm}$ over bordet. & Til alle test hvor der tabes en bold, tabes den fra $30\si{cm}$ & Systemet fungere & \\ \hline 
K\ref{kravspec:req:max-cup-offset} & Systemet skal detektere placering af en \textbf{Cup} når afstanden mellem centrum af \textbf{Cup} og centrum af \textit{\textbf{Cupholder}} er under 10mm & Der placeres en \textbf{Cup} hvis centrum er 10mm (målt med et skydelære med en opløsning på 1/2 mm) fra centrum af \textit{\textbf{Cupholder}}& Farven for den gældene \textbf{\textit{Cup light}} skifter & Som forventet\\ \hline
K\ref{kravspec:req:ball-dispenser-full-amount} & \textit{\textbf{Ball dispenser}} skal kunne indeholde 14 \textbf{Balls}. & Der påfyldes én bold af gangen, indtil der ikke kan være flere bolde. Det tælles hvor mange der kan være. & 14 eller flere \textbf{Balls} & Der kan være 4\\ \hline
\caption{Accepttestspecifikation for Fysiske parametre}
\label{tab:IkkeFunktFysiske}
\end{longtable}



\begin{longtable}{|L{0.08\textwidth}|L{0.3\textwidth}|L{0.3\textwidth}|L{0.2\textwidth}|L{0.2\textwidth}|}
\hline
\textbf{Krav ID} & \textbf{Krav} & \textbf{Test} & \textbf{Forventet resultat} & \textbf{Faktisk resultat} \\ \hline

K\ref{kravspec:req:wifi-name}& Systemet skal hoste et wifi netværk med SSID ''Beer\_Pong\_Table'' og adgangskoden ''beerpong''. & Det undersøges med en  iPhone 7 hvilke wifi netværk der eksistere, og hvis et netværk med SSID ''Beer\_Pong\_Table'' eksistere tastes koden ''beerpong'' ind & SSID: ''Beer\_Pong\_Table'' og adgangskode: ''beerpong'' & Som forventet \\ \hline
K\ref{kravspec:req:led-same-color} & De fem LED'er i \textit{\textbf{Cup light}} til en \textit{\textbf{Cupholder}} skal lyse med den samme farve. & Det observeres hvilken farve LED'erne lyser med & De lyser med samme farve & Der er kun en led\\ \hline
K\ref{kravspec:req:color-count} & Et enkelt \textit{\textbf{Cup light}} skal kunne lyse med mindst 10 forskellige farver & Spillet startes flere gange hvor der på WebPage indstilles forskellige farver, og de forskellige farver observeres. & Der er mindst 10 forskellige farver & Der er gul, blå, grøn, rød, lilla, cyan, hvid, orange-gul-grøn, blåligt hvidt og pinkt hvidt.\\ \hline
K\ref{kravspec:req:color-control} & Hvert \textit{\textbf{Cup light}} skal kunne styres individuelt & Det observeres om kun en \textit{\textbf{Cup light}} ændrer sig når en \textbf{Cup} placeres & Kun en farve \textbf{\textit{Cup light}} skifter & Som forventet\\ \hline 
K\ref{kravspec:req:missing-cup-color} & Et \textit{\textbf{Cup light}} skal i UC1 lyse gul når der ikke er en \textbf{Cup} i den tilhørende \textit{\textbf{Cupholder}}  & Det observeres som en del af testen af UC1, når der ikke står en \textbf{Cup}.& \textbf{\textit{Cup light}} lyser gult & Som forventet, lidt grøn\\ \hline
K\ref{kravspec:req:placed-cup-color} & Et \textit{\textbf{Cup light}} skal i UC1 lyse blå når der er en \textbf{Cup} i den tilhørende \textbf{Cup} holder & Det observeres som en del af testen af UC1, når der står en \textbf{Cup} & \textbf{\textit{Cup light}} lyser blåt & Som forventet \\ \hline
K\ref{kravspec:req:status-colors} & Lysdioderne på \textit{\textbf{Ball dispenser status lights}} skal lyse i hver deres farve, rød og grøn & Det observeres som en del af testen af UC4 farverne på led'erne & Den ene led er grøn og den anden er rød & Som forventet \\ \hline
K\ref{kravspec:req:status-empty-light} & Den røde lysdiode i \textit{\textbf{Ball dispenser status lights}} skal lyse, når der er færre end to \textbf{Balls} tilbage i \textit{\textbf{Ball dispenser}} & Det observeres som en del af testen af UC4 om den røde led lyser når der er færre end to \textbf{Balls} & Den røde led lyser rød & ok, men den lyser også når der er 2 \\ \hline
K\ref{kravspec:req:status-full-light}& Den grønne lysdiode i \textit{\textbf{Ball dispenser status lights}} skal lyse, når \textit{\textbf{Ball dispenser}} er fuld af \textbf{Balls} som specificeret i krav K\ref{req:ball-dispenser-full-amount} & Det observeres som en del af testen af UC4 om den grønne led lyser når der er 14 \textbf{Balls} i \textbf{\textit{Ball dispenser}} & Den grønne led lyser & Den grønne led lyser når der er 4.   \\ \hline
K\ref{kravspec:req:webpage-address} & WebPage er en hjemmeside med statisk IP-adresse 10.9.8.2., som er hostet af systemet & Der logges på det hostede wifi netværk og siden 10.9.8.2 tilgåes. & Der er en hjemmeside & Som forventet\\ \hline
K\ref{kravspec:req:webpage-language} & Alt tekst på WebPage skal være på sproget engelsk. & Sproget på hjemmesiden observeres & Hjemmesiden er på engelsk & Som forventet\\ \hline
K\ref{kravspec:req:webpage-info-type} & På WebPage skal der for hvert af de to hold indtastes følgende \textit{\textbf{spiller oplysninger}}: \textit{teamname}, \textit{username1} og \textit{username2}. & Det observeres hvilke indtastningsmuligheder der er. &  Der er for begge hold mulighed for at indtaste \textit{teamname}, \textit{username1} og \textit{username2}&  So forventet\\ \hline
K\ref{kravspec:req:webpage-string-length} & Holdnavne og brugernavne indtastet på WebPage skal være tekststrenge, med en længde på mellem 1 og 15 karakterer. & Som en del af testen for UC1, benyttes der strenge på 1 karakter og strenge på 15 karakterer. Det observeres hvilke strenge der vises på \textbf{\textit{Display}} efter der trykkes 'Start' & De rigtige tekststrenge vises på \textbf{\textit{Display}} & Intet display\\ \hline
K\ref{kravspec:req:webpage-color-select} & På WebPage skal der for hvert af de to hold kunne vælges følgende \textit{\textbf{spiller oplysninger}}: \textbf{holdfarve}, ved at justere intensiteten af farverne rød, grøn og blå i et RGB farveskema. & Som en del af testen for UC1, indstilles forskellige farver. & De valgte farver vises på de forskellige \textbf{Cup lights} & Som forventet\\ \hline
K\ref{kravspec:req:webpage-color-size} & Der skal kunne vælges mindst 10 forskellige farver på WebPage. & Udføres som en del af testen af K\ref{kravspec:req:color-count} & Der kan vælges 10 eller flere farver & Som forventet, og der kan vælges mange flere\\ \hline
K\ref{kravspec:req:GUI-language} & Alt tekst på GUI skal være på sproget engelsk. & Sproget på GUI'en observeres & Der benyttes engelsk &  Intet display\\ \hline
K\ref{kravspec:req:GUI-info-type} & På GUI skal der for hvert af de to hold vises \textit{\textbf{spiller oplysninger}}: \textit{teamname}, \textit{username1} og \textit{username2}. & Som en del af testen UC1 og UC2 observeres det hvilke oplysninger der vises & Der vises for begge hold følgende oplysninger: \textit{teamname}, \textit{username1} og \textit{username2}& Intet display \\ \hline


\caption{Accepttestspecifikation for User interface}
\label{tab:IkkeFunktFysiske}
\end{longtable}



\begin{longtable}{|L{0.08\textwidth}|L{0.3\textwidth}|L{0.3\textwidth}|L{0.2\textwidth}|L{0.2\textwidth}|}
\hline
\textbf{Krav ID} & \textbf{Krav} & \textbf{Test} & \textbf{Forventet resultat} & \textbf{Faktisk resultat} \\ \hline

K\ref{kravspec:req:ball-response-time} & Den første \textbf{Ball} skal have forladt \textit{\textbf{Ball dispenser}} indenfor 5s efter der indsættes en \textbf{Coin}. & Der optages en video hvor der indsættes en \textbf{Coin} og der ventes til at en den første \textbf{Ball} er leveret. Ud fra videoen beregnes det hvor lang tid det tog.  & Under 5s & \\ \hline
K\ref{kravspec:req:won-response-time} & Når den sidste \textbf{Cup} er fjernet fra sin \textbf{\textit{Cupholder}}, viser \textbf{\textit{Display}} hvem der har vundet indenfor 5s. & Der optages en video hvor der fjernes den sidste \textbf{Cup} og der ventes til at \textbf{\textit{Display}} viser at et hold har vundet. Ud fra videoen beregnes det hvor lang tid det tog. & under 5s & Intet display\\ \hline
K\ref{kravspec:req:waterproof} & Systemet skal være stænktæt. Dvs. øl, sodavand og andet væske ikke kan komme til systemet indre sensorer eller computer. & Det observeres hvad der sker når der spildes øl på bordet & Systemet virker stadig & Testes ikke\\ \hline

K\ref{kravspec:req:startup-turnon-time} & Systemets \textbf{\textit{start op}} tid skal være mindre end 10s. & Der optages en video, hvor systemet med en fuld bolddispenser forbindes til en spændingsforsyning og der ventes til at \textbf{\textit{Display}} viser ''Klar til spil'' og den grønne LED på \textbf{\textit{Ball dispenser status lights}} er tændt. Ud fra videoen beregnes det hvor lang tid det tog. & Under 10s & Testes ikke \\ \hline
K\ref{kravspec:req:startup-turnoff-time} & Systemets \textbf{\textit{slukke}} tid skal være mindre end 10 sekunder. & Der optages en video, hvor systemet med en fuld bolddispenser afbrydes spændingsforsyningen og der ventes til at \textbf{\textit{Display}} ikke viser noget og den grønne LED på \textbf{\textit{Ball dispenser status lights}} slukker og \textbf{\textit{Cup lights}} slukker. Ud fra videoen beregnes det hvor lang tid det tog. & Under 10s  & Testes ikke  \\ \hline
K\ref{kravspec:req:remove-cup-response-time} & \textit{\textbf{Display}} skal vise at en person har fjernet et \textbf{Cup} fra en \textit{\textbf{Cupholder}} indenfor 500ms. & Der optages en video hvor der en \textbf{Cup} (som ikke er den sidste) og der ventes til at \textbf{\textit{Display}} viser at \textbf{Cup} er fjernet. Ud fra videoen beregnes det hvor lang tid det tog. & Under 500ms&  Intet display: Testes ikke \\ \hline
K\ref{kravspec:req:max-start-time} & Fra der trykkes 'Start' på WebPage til at \textbf{\textit{Cup light}} opdateres må der som maksimum gå 1s. & Dette testes som en del af UC3. Det optages en video hvor der trykkes 'Start', hvorefter der ventes på at \textbf{\textit{Cup light}} opdateres. Ud fra videoen beregnes det hvor lang tid det tog & Under 1s & \\ \hline
\caption{Accepttestspecifikation for Ydeevne}
\label{tab:IkkeFunktYdeevne}
\end{longtable}


\begin{longtable}{|L{0.08\textwidth}|L{0.3\textwidth}|L{0.3\textwidth}|L{0.2\textwidth}|L{0.2\textwidth}|}
\hline
\textbf{Krav ID} & \textbf{Krav} & \textbf{Test} & \textbf{Forventet resultat} & \textbf{Faktisk resultat} \\ \hline

K\ref{kravspec:req:placed-succes-rate} & Det skal detekteres 99 ud af 100 gange at der placeres en \textbf{Cup} i en \textit{\textbf{Cupholder}}. & Imens \textbf{\textit{spillet er i gang}} placeres der en \textbf{Cup} på en \textbf{\textit{Cupholder}} og det observeres på \textbf{\textit{Display}} om dette detekteres. Herefter fjernes \textbf{Cup}. Det gentages 100 gange. & Mindst 99 ud af 100 gange viser \textbf{\textit{Display}} at en \textbf{Cup} er placeret & Intet display, derfor observeres Cup light istedet for. 100 ud af 100 gange\\ \hline
K\ref{kravspec:req:removed-succes-rate} & Det skal detekteres 99 ud af 100 gange at der løftes en \textbf{Cup} fra en \textit{\textbf{Cupholder}}. & Dette udføres samtidig med testen for K\ref{kravspec:req:placed-succes-rate}. & Mindst 99 ud af 100 gange viser \textbf{\textit{Display}} at en \textbf{Cup} er fjernet & Intet display, derfor observeres Cup light istedet for. 100 ud af 100 gange\\ \hline
K\ref{kravspec:req:dropped-succes-rate} & Det skal detekteres 95 ud af 100 gange at en \textbf{Ball} rammer i  en \textbf{Cup} som står i en \textit{\textbf{Cupholder}}. & Imens \textbf{\textit{spillet er i gang}} tabes der en \textbf{Ball} fra 30 cm højde, og det observeres hvor mange gange \textbf{\textit{Cup light}} begynder at blinke. Herefter fjernes \textbf{Cup} og \textbf{Ball} fra \textbf{Cup}. Det gentages 100 gange. & Mindst 95 ud af 100 gange blinker \textbf{\textit{Cup light}} & 86 uf af 100 gange blinker \textbf{\textit{Cup light}}\\ \hline
K\ref{kravspec:req:hour-false-placed} & I en periode på 1 time må der højest være 1 falsk detektering af placering af en \textbf{Cup} på tom \textit{\textbf{Cupholder}} & (Dette tester også K\ref{kravspec:req:hour-false-removed}) Imens \textbf{\textit{spillet er i gang}} placeres der 3 \textbf{Cups} på 3 af de 6 \textbf{\textit{Cupholders}} i en \textbf{\textit{Playerside}}. Herefter observeres der på \textbf{\textit{Cup light}}, hvor mange gange det detekteres at en \textbf{Cup} placeres & Maks 1 gang & 0 gange \\ \hline

K\ref{kravspec:req:hour-false-removed}& I en periode på 1 time må der højest være 1 falsk detektering af løfting af en \textbf{Cup} fra \textit{\textbf{Cupholder}} hvor der står en \textbf{Cup}& Dette testes som en del af testen af K\ref{kravspec:req:hour-false-placed}. Der observeres på \textbf{\textit{Cup light}} hvor mange gange det detekteres at en \textbf{Cup} fjernes & Maks 1 gang & 0 gange\\ \hline

K\ref{kravspec:req:ball-hit-sensor-false-placed} & Når en \textbf{Ball} rammer en \textit{\textbf{Cupholder}} og hopper væk igen 100 gange (uden nogen \textbf{Cup}) må der højest ske 1 falsk detektering af at der placeres en \textbf{Cup} i den givne \textit{\textbf{Cup holder}}& En bold kastes hen mod en \textbf{\textit{Playerside}} 100 gange og det observeres på \textbf{\textit{Cup light}} hvor mange gange der detekteres at en \textbf{Cup} er placeret & Maks 1 gang & 0 gange\\ \hline
K\ref{kravspec:req:inserted-coin-succes-rate} & Det skal detekteres mindst 98 ud af 100 gange at der indsættes en 5 krone i bolddispenser. & Der indsættes en mønt og det observeres om \textit{\textbf{Cup light}} tænder. Systemet genstartes og testen udføres 100 gange. Det tælles hvor mange gange \textit{\textbf{Cup light}} tænder& Mindst 98 ud af 100 gange & 38 ud af 100 gange \\ \hline
K\ref{kravspec:req:inserted-wrong-coin-false-detect} & Det skal detekteres højst 2 ud af 100 gange at der indsættes en 5 krone i bolddispenser når der indsættes enhver anden dansk \textbf{Coin}. & Der indsættes en dansk 20 krone 100 gange, og det observeres om \textbf{\textit{Cup light}} tænder. Hvis \textbf{\textit{Cup light}} tændes genstartes systemet. Det tælles hvor mange gange \textbf{\textit{Cup light}} tænder & Maks 2 ud af 100 gange & Ikke testet \\ \hline
K\ref{kravspec:req:hour-false-coin} & I en periode på 1 time må der højest være 1 falsk detektering af indsættelse af en dansk 5 krone i bolddispenser & Der ventes i en time og det observeres hvor mange gange \textbf{Cup light} tænder og der derfor er detekteret en mønt. Hvis dette sker genstartes systemet. Det tælles hvor mange gange \textbf{\textit{Cup light}} tænder & Maks 1 gang & Ikke testet \\ \hline

K\ref{kravspec:req:ball-delivery-error-count} & Der må højst ske en fejl ved levering af 2 \textbf{Balls} (dvs. der leveres færre eller flere end 2 \textbf{Balls}) 1 ud af 50 gange.  & Dette testes som en del af testen for K\ref{kravspec:req:inserted-coin-succes-rate}. Der tælles hvor mange gang der leveres to bolde, når der indsættes en mønt. Hvis \textbf{\textit{Cup light}} ikke tændes, tælles det ikke med som en fejl at der ikke leveres nogle bolde & Maks 1 ud af 50 gange & 0 ud af 50 gange\\ \hline

K\ref{kravspec:req:webpage-error-rate}& Mindst 49 ud af 50 gange, hvor der trykkes 'Start' på WebPage, skal de rigtige \textit{\textbf{spiller oplysninger}} fremgå af display og \textit{\textbf{Cup lights}}. & Systemet startes og der placeres \textbf{Cups} på alle \textbf{\textit{Cup holders}}, herefter indtastes informationer på hjemmeside og det observeres om de rigtige oplysninger bliver vist på \textbf{Display}. Dette gentages 50 gange og der tælles hvor mange gange der vises de rigtige oplysninger & Mindst 49 ud af 50 gange & Intet display.  men cup light observeres. 50 ud af 50 gange\\ \hline

\caption{Accepttestspecifikation for Pålidelighed}
\label{tab:IkkeFunktReliability}
\end{longtable}

\end{document}