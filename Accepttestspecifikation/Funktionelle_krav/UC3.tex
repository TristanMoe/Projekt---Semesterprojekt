\documentclass[Accepttestspecifikation/Accepttest_Main.tex]{subfiles}

\begin{document}
\subsection{Accepttest - UC3}
\begin{table}[H]
\centering
\resizebox{\textwidth}{!}{%
\begin{tabular}{|l|l|l|l|l|}
\hline
\textbf{UC under test} & \multicolumn{4}{l|}{UC3 - Stop game}                                                                                                                                                                                                                                                      \\ \hline
\textbf{Scenarie}      & \multicolumn{4}{l|}{Hovedscenarie}                                                                                                                                                                                                                                                        \\ \hline
\textbf{Prækondition}  & \multicolumn{4}{l|}{Sidste kop af ramt på en af spillersiden}                                                                                                                                                                                                                             \\ \hline
                       & \textbf{Handling}                                                                                  & \textbf{\begin{tabular}[c]{@{}l@{}}Forventet \\ resultat\end{tabular}}                   & \textbf{\begin{tabular}[c]{@{}l@{}}Faktisk \\ resultat\end{tabular}} & \textbf{Kommentar} \\ \hline
\textbf{Punkt 1}       & \begin{tabular}[c]{@{}l@{}}Brugeren fjerner \\ det sidste kop på \\ den ene side af bordet.\end{tabular}              & \begin{tabular}[c]{@{}l@{}}Bordet slukker for \\ LED'er under \\ kopperne.\end{tabular} &                                                                      &                    \\ \hline
\textbf{Punkt 2}       & \begin{tabular}[c]{@{}l@{}}Kop-lys bliver opdateret \\ så de viser\\ hvilken playerside \\ vandt og tabt.\end{tabular} & \begin{tabular}[c]{@{}l@{}} Koplys for det playerside\\ hvis hold har vundet\\  lyser med sejrsfarver\\ og modstændernes koplys\\ lyser med\\ nederlagsfarver.\end{tabular}      &                                                                      &                    \\ \hline
\textbf{Punkt 3}       & \begin{tabular}[c]{@{}l@{}}Display på bordet \\opdateres så\\ med navnet på det\\ vindende hold. \end{tabular} & \begin{tabular}[c]{@{}l@{}}Display på bordet\\ viser navnet på\\ det hold der vandt.\end{tabular}      &                                                                      &                    \\ \hline
\end{tabular}
}
\end{table}


\end{document}