\documentclass[Rapport/Rapport_main.tex]{subfiles}
\begin{document}
\section{Konklusion}
%Som start på konklusion
I dette afsnit konkluderes der på \textbf{Projekt Formuleringen} i forhold til, hvad der er lykkedes og hvad der ikke er. Derefter konkluderes der på den proces, der har været undervejs i projektet.\\\\
%Konklusion på Projekt formulering
Målet med dette projekt er som sagt en funktionel prototype af et interaktivt Beer Pong bord. En prototype er implementeret, der består af den ene side af bordet, samt en bolddispenser med møntindkast. Disse delsystemer er implementeret med en PSoC platform\cite{psoc5lp}. Bolddispenseren detekterer ved sensorer indkast af mønter og kapacitet af bolde. Den anvender desuden en stepper-motor til både indsamling af mønter og dispensering af bolde.\\
Spillersiden har 6 kopholder, der anvender sensorer til detektering af anbringelse, fjernelse og hændelsen af en bordtennisbold, der rammer i en kop. Derudover så reagerer lys i kopholderne udfra førnævnte detekteringer.\\
Da både bolddispenser og spillerside anvender sensorer og aktuatorer kan det konkluderes, at kravet omkring dette er opfyldt. \\
Gennem I2C-kommunikationsprotokol\cite{i2c_protocol} kommunikerer delsystemerne med en RPi Zero W \cite{rpi_webpage}. Da RPi Zero W er en indlejret Linux platform kan det konkluderes, at kravet om en indlejret Linux platform opfyldes.\\
Det kan konkluderes, at RPi hoster en hjemmeside, hvorpå det er muligt at indtaste holdnavne, brugernavne og vælge holdfarver. Gennem hjemmesiden er det muligt at styre farven på lyset i kopholderne, men det er ikke muligt at vise holdnavne og brugernavne på et Display.\\\\
%Proces relateret konklusion
Måden de enkelte dele blev testet på var ved, at komponenter gradvist blev samlet til undersystemer indtil det komplette top-niveau system var samlet - denne tilgang kendes også som Bottom-Up. I starten følte vi denne tilgang til integration af komponenterne gav et anskueligt overblik over projektets status, men da den endelige integrationstest blev lavet, opstod der flere problemer, som var ukendt ved de isolerede integrationstest. Det måtte opgives at løse de nye opståede problemer, da vi ikke havde tildelt nok tid til integrationstest og nogle delsystemer ikke var klar før til sidst i processen (Se figur \ref{integration:fig:Bottom-Up}. Det var angivet i tidsplanen, se figur "XX", at alt integrationstest skulle være udført inden uge 50. Der blev desuden også sat en deadline d. 14-12-2018 til integrationtest, da accepttesten skulle afholdes samme dag. Her blev integration af Display i RPiApp forkastet pga. mangel på ressourcer og tid, og dette medførte som sagt at produktet ikke levede op til de definerede krav. \\\\
Udviklingsprocessen for dette projekt er gennemført ved hjælp af en iterativ tilgang til ASE-udviklingsmodellen. Til forskel for det forrige semester, hvor ASE-modellen kun var i en iterativ fase ved design / implementering. Denne tilgang medførte uoverskuelige konsekvenser ved ændring af kravspecifikation, arkitektur mv. Den nye arbejdsprocess, med de indførte iterative metoder i form af Scrum og den rent iterative ASE-model, gjorde det lettere at have nye tilføjelser til projektet.  

\end{document}