\documentclass[Rapport/Rapport_main.tex]{subfiles}
\begin{document}
\section{Konklusion}
%Som start på konklusion
I dette afsnit konkluderes der på \textbf{Projektformuleringen} i forhold til, hvad der er lykkedes, og hvad der ikke er. Derefter konkluderes der på den proces, der har været undervejs i projektet.\\\\
%Konklusion på Projekt formulering
En funktionel prototype af et interaktivt Beer Pong bord er implementeret. Der er kun implementeret den ene af siderne af bordet, desuden er en bolddispenser med møntindkast implementeret. Disse delsystemer er implementeret med en PSoC platform\cite{psoc5lp}. Bolddispenseren detekterer ved sensorer indkast af mønter og kapacitet af bolde. Den anvender desuden en stepper-motor til både indsamling af mønter og dispensering af bolde.\\
Spillersiden har 6 kopholdere, der anvender sensorer til detektering af anbringelse, fjernelse og hændelsen af en bordtennisbold, der rammer i en kop. Derudover så reagerer lys i kopholderne ud fra førnævnte detekteringer.\\
Da både bolddispenser og spillerside anvender sensorer og aktuatorer, kan det konkluderes, at kravet omkring dette er opfyldt. \\
Gennem I2C-kommunikationsprotokol\cite{i2c_protocol} kommunikerer delsystemerne med en RPi Zero W \cite{rpi_webpage}, der er en indlejret Linux platform. Derfor kan der konkluderes, at kravet om en indlejret Linux platform opfyldes.\\
Det kan konkluderes, at RPi hoster en hjemmeside, hvorpå det er muligt at indtaste holdnavne, brugernavne og vælge holdfarver. Gennem hjemmesiden er det muligt at styre farven på lyset i kopholderne. På bordet er det ikke muligt at vise holdnavne og brugernavne på et Display.\\\\
%Proces relateret konklusion
Der kan konkluderes, at der er anvendt ASE-udviklingsmodellen på en iterativ måde. Der er anvendt Scrum som tilgang til den iterative udvikling, hvilket medførte en mindskning af konsekvenser ved ændring af kravspecifikation og arkitektur. \\
Det kan desuden konkluderes, at estimeringen af varigheden af udviklingsprocessen har været mangelfuld i det, der var mangel på tid til integration af delsystemerne - her måtte flere elementer forkastes pga. tidsmangel og ressourcer.   
\end{document}