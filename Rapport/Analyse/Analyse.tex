\documentclass[Rapport/Rapport_main.tex]{subfiles}
\begin{document}
\section{Analyse}\label{sec:rapport_analyse}
%Her beskrives analysearbejdet, dvs. de overvejelser om mulige løsninger I har haft, de løsninger I har valgt og begrundelsen herfor. De grundlæggende valg af hardware- og softwaremæssige komponenter, som er kritiske for realisering af systemet, beskrives her. For at træffe et valg kan der analyseres og diskuteres forskellige løsninger mht. til ydeevne, pris, leveringstid og forhåndskendskab. Disse kan med fordel opstilles i tabelform. Herefter kan der træffes valg, som danner et ”proof of concept” for de forskellige dele af projektet. Der skal selvfølgelig argumenteres for disse valg. Der kan med fordel refereres til detaljer i dokumentet ”Analyse” i projektets bilag.
I dette afsnit beskrives resultater af analysen fra bilag \textbf{Analyse}.
\subsection{System opdeling}
For at opnå lav kobling laves et distribueret system, der består af et delsystem for hver spillerside, for bolddispenser med møntindkastning og for en systemcontroller.
Der kan herefter kigges på CPU'erne, der skal anvendes til delsystermerne. Argumentationen for valgene kan ses i \textbf{Analyse} dokumentet i afsnit \fullref{analyse:sec:sys_opdeling}, men her præsenteres bare resultaterne. Til spillersiderne anvendes en PSoC 5LP Prototyping kit\cite{psoc5lp}. Til bolddispenser anvendes en PSoC 5LP Prototyping kit\cite{psoc5lp}. Til systemcontroller anvendes en Rasberry Pi Zero W med indlejret Linux\cite{rpi_webpage}.

\subsection{Kommunikation mellem delsystemer}
Kommunikationen mellem delsystemerne skal gå gennem systemcontroller. Af denne grund kan den benævnes som \textit{master} og de andre delsystemer som \textit{slaves}.\\\\
Overvejelsen omkring \textbf{Kommunikations protokol} går ud fra, hvilke protokoller er tilgængelige på de valgte CPU'er og hvilke, der er stiftet bekendtskab med på 3. semester. Derfor er overvejelsen mellem \textbf{SPI} og \textbf{I2C}. Selvom SPI er hurtigst, så er det vigtigere at have færre linjer mellem \textit{master} og \textit{slaves}, og I2C kan desuden anvendes med en 5V PSoC og 3.3V RPi uden yderligere modificering af PSoC. Dette og alle overvejelser er uddybet videre i Analyse dokumentet i afsnit \fullref{analyse:sec:comm_analyse}.\\

Der kan nu bygges en arkitektur for systemet.

\end{document}