\documentclass[Rapport/Rapport_main.tex]{subfiles}
\begin{document}
\section{Analyse}\label{sec:rapport_analyse}
%Her beskrives analysearbejdet, dvs. de overvejelser om mulige løsninger I har haft, de løsninger I har valgt og begrundelsen herfor. De grundlæggende valg af hardware- og softwaremæssige komponenter, som er kritiske for realisering af systemet, beskrives her. For at træffe et valg kan der analyseres og diskuteres forskellige løsninger mht. til ydeevne, pris, leveringstid og forhåndskendskab. Disse kan med fordel opstilles i tabelform. Herefter kan der træffes valg, som danner et ”proof of concept” for de forskellige dele af projektet. Der skal selvfølgelig argumenteres for disse valg. Der kan med fordel refereres til detaljer i dokumentet ”Analyse” i projektets bilag.
I dette afsnit beskrives resultater af analysen fra bilag \textbf{Analyse}. Der laves også en opsummering af de vigtigste tanker og overvejelser.
\subsection{System opdeling}
En vigtig overvejelse gik på, hvilken opdeling af systemet, der ønskes. Der er forskellige muligheder her, hvor overvejelsen går på enten et centraliseret system eller et system i moduler. Det centraliserede system har fordelen at det er billigere, hvor fordelen ved flere moduler er lavere kobling og højere samhørighed. Løsningen der vælges er modul opdelingen, da den også vil gøre det nemmere at forbedre, teste og udskifte de enkelte dele.\\\\
Nu hvor en opdeling er valgt, er det værd at identificere de enkelte dele og hvad de indeholder i forhold til sensorer og aktuatorer. Der kan identificeres de 2 spillersider som delsystemer, da der skal detekteres og reageres på \textit{Game Events} på begge sider. Dette udreder at der skal anvendes både en sensor og aktuator på hver side.\\ 
Der kan også identificeres et delsystem til bolddispenser og møntindkast, da de en detektering af en mønt skal udløse en dispensering af to bolde. Ved en detektering af en mønt skal denne indsamles. Der skal detekteres boldniveauet i bolddispenseren, hvilket brugeren skal meddeles om. Dette udreder, at der her skal bruges en sensor til detektering af mønt, en aktuator til dispensering af bolde, en aktuator til indsamling af mønt, en sensor til detektering af boldniveau og en aktuator til at angive boldniveauet.\\
For at opnå den lave kobling skal ingen af de allerede angivende delsystemer stå for kontrol af stadier af hele systemet og kommunikationen skal heller ikke foregå internt imellem dem. Derfor identficeres det, at der skal være en hjerne i systemet. Denne hjerne skal kunne skrive til et display og hoste og kommunikere med en webserver. Det udreder, at der skal anvendes et display som aktuator.\\\\
Det er herefter interessant at kigge på CPU'erne, der skal anvendes til delsystermerne. Argumentationen for valgene kan ses i \textbf{Analyse} dokumentet i afsnit \fullref{analyse:sec:sys_opdeling}, men her præsenteres bare resultaterne. Til spillersiderne anvendes en PSoC 5LP CY8C58LP \cite{psoc5lp}. Til bolddispenser anvendes en PSoC 5LP CY8C58LP \cite{psoc5lp}. Til hjernen af systemet anvendes en RPi Zero W med indlejret Linux\cite{rpi_webpage}, og vil fremover benævnes RPi.

\subsection{Kommunikation mellem delsystemer}
Kommunikation mellem delsystemerne blev nævnt, hvor at der ønskedes en kommunikation, der gik gennem RPi. Af denne grund kan RPi benævnes som \textit{master} og de andre delsystemer som \textit{slaves}.\\\\
Overvejelsen omkring \textbf{Kommunikations protokol} går ud fra, hvilke er tilgængelige på de valgte komponenter og hvilke, der er stiftet bekendtskab med på 3. semester. Derfor er overvejelsen mellem \textbf{SPI} og \textbf{I2C}. Selvom SPI er hurtigst, så er det vigtigere at have færre linjer mellem \textit{master} og \textit{slaves}, og I2C kan desuden anvendes med en 5V PSoC og 3.3V RPi uden yderligere modificering af PSoC. Dette og alle overvejelser er uddybet videre i Analyse dokumentet i afsnit \fullref{analyse:sec:comm_analyse}.

\end{document}