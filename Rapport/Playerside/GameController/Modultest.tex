\documentclass[Rapport/Playerside/GameController/GameController.tex]{subfiles}
\label{sec:playerside_GameController_modultest}
\begin{document}
\subsubsection{Modultest}
I modultesten for GameController klassen blev klassen RPI\_IF brugt, da den allerede var færdig på tidspunktet af testen. Der benyttes stubs til metoderne i CupLight\_IF og CupSensor\_IF. For at simulere en opdatering af CupStatus og HitStatus blev de to metoder updateCupStatus og updateHitStatus kaldt via UART. Det vil sige, at en UART komponent blev tilføjet til projektet, som skulle tage et tastetryk fra tastaturet, hvorved en kombination af de funktioner ville blive kaldt. Analog discovery blev også brugt til at sende beskeder til PSoC'en Gamecontroller. \\\\
I afsnit \ref{modultest:sec:GameController_modultest_bilag} i bilaget \textbf{modultest}  kan en mere uddybende test ses.

\end{document}