\documentclass[Rapport/Playerside/GameController/GameController.tex]{subfiles}
\label{sec:playerside_GameController_design}
\begin{document}
\subsubsection{Design}
I dette afsnit beskrives design for GameController klassen på playerside. Denne klasse er controller klassen for playerside, og indeholder alt den logik, der skal til, for at de 3 boundary klasser RPI\_IF, CupSensor\_IF og CupLight\_IF fungerer i et samlet system. Under designet af denne klasse har det været vigtigt at følge sekvensdiagrammet og statediagram i arkitekturet og bruge de funktioner, der har været defineret i de forskellige boundaryklasser. En del af designet af klassen bestod i også alt lave stubs for funktionerne i de 3 boundaryklasser, så det eneste der skulle gøres i en videre integrering var at slette implementeringen af stubs og bruge de rigtige funktioner. UART er generelt brugt rigtig meget til at udskrive information om de forskellige funktioner. På denne måde var systemet let at debugge. En funktion i gamecontrolleren, der er interessant at snakke om er blinkefunktionen i systemet, som denne klasse skal styre. Her bliver en timer i PSoC komponent katalog brugt, som løber konstant, når den startes, og skal interrupte hvert halve sekund. Her bliver en funktion interrupt\_blink, der skal kontrollere de 6 LED'er, når de  skal blinke. Timer komponenten brugt kan ses i figur \ref{fig:Timer}.
\begin{figure}
    \centering 
    \includegraphics[width=0.5\linewidth]{Softwaredesign/GameController/graphic/gamecontroller_timer.PNG}
    \caption{Timer brugt til implementering af blink funktion}
    \label{fig:Timer}
\end{figure}
For en mere uddybende beskrivelse af design for GameController klassen, heri de forskellige funktioner, så se afsnit \fullref{sec:GameController_design_bilag} i bilaget.
\end{document}
