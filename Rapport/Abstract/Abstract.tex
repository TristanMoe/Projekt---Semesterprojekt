\documentclass[Rapport/Rapport_main.tex]{subfiles}
\begin{document}

\section{Resume}
I dette projekt anvendes en Scrum baseret udviklingsproces til at designe og teste et Beer Pong bord. Det er valgt ikke at realisere et fuldt bord, men i stedet en fungerende prototype af én side af bordet. Der anvendes sensorer til at registrere tilstedeværelsen af en kop og den hændelse, at en bordtennisbold rammer ned i koppen. Dette påvirker farven på det lys, som er lokaliseret under kopperne. Bordet er desuden forsynet med en bolddispenser, som via sensorer kan detektere mønter og bordtennisbolde. Stepper-motorer anvendes til at frigive mønter og dispensere bolde. Disse dele af systemet styres af to PSoC platforme, som anvender I2C-protokollen til at kommunikere med en Raspberry Pi Zero W. Denne hoster en hjemmeside, hvorpå brugerne kan indtaste navne og holdfarver. Farverne afspejles i lyset under kopperne, mens navne og score skulle fremgå af et display. Store dele af systemet er implementeret og integreret, og en accepttest er udført, men ikke alle krav er opfyldt. Yderligere arbejde er påkrævet, hvis display skal integreres, eller det fulde bord skal realiseres.

\section{Abstract}
In this project a Scrum based development process is used to design and test a Beer Pong Table. Instead of a full table it was decided to produce a functioning prototype of only one side of the table. Sensors are used to detect when a ping-pong ball lands in a cup, and also whether or not a cup is present. These events affect the color of the lights that are located under the cups. The table also has a ball dispenser which uses sensors to detect coins and ping-pong balls. Stepper motors are used in order to release coins and dispense balls. These parts of the system are controlled by two PSoC platforms. The PSoC platforms utilize an I2C protocol, in order to communicate with a Raspberry Pi Zero W. The Raspberry Pi hosts a website, where the users can enter names and colors. The colors are depicted in the lights under the cups, and the names and scores are supposed to appear on a display. Large parts of the system are implemented and integrated, and acceptance testing was performed. Not all requirements are met and thus further work is necessary in the process of integrating the display and realising the full table. 
\end{document}