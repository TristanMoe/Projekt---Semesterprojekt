\documentclass[Rapport/RPi/RPI.tex]{subfiles}
\begin{document}
\subsubsection{Softwaredesign}\label{sec:I2C_interruptDriver_design}
I dette afsnit beskrives software design for I2C\_interruptDriver. Driveren er lavet, som en char driver. Driveren har til formål, at skrive til og læse fra de forskellige PSoC's i systemet ved hjælp af I2C. Der laves til dette formål en platform driver til dette formål. For at være i stand til at bruge I2C i vores driver har vi gjort brug af biblioteket i2c.h. For at bruge en I2C bus på vores RPI'en bruger vi funktionen i2c\_get\_adapter, hvorfra et struct for adapteren fås. Man kan på denne adapter tilføje 3 klienter, som er vores 3 PSoC's. Der skal for disse 3 enheder laves 3 structs af typen I2C\_board\_info, hvori et navn og addressen på I2C enheden gives. Informationerne i dette struct bruges til at lave et struct af typen I2C\_Client.  Dette struct laves ved hjælp af funktionen I2C\_create\_device. I driveren laves der en node for alle disse klienter, og et struct(i2c\_interrupt\_device) laves med alle nødvendig informationer for klienterne, såsom navn og minor nummer mm. Da driveren er en platform driver er der en probe og remove funktion. I probe vil alt initiering af adapter og alle I2C klienterne finde sted, samt initiering af gpio's brugt til interrupts. Information omkring gpio'er brugt til interrupts er defineret i device tree. I probe vil disse informationer trækkes ud. I driverens open funktion vil interrupts initieres og ligeledes frigives i release. Driverens read funktion skal læse fra alle PSoC's(en for hver node). Når I2C\_interrupt\_read kaldes vil den med det samme lægge sig til at sove. Dette gør den for efterfølgende at skulle vækkes af et interrupt fra en af PSoC's. Når den vækkes, vil information fra PSoC's læses ved hjælp af funktionen I2C\_master\_recv for den klient(PSoC), der skaber interruptet. Informationen læst vil herefter kopieres til userspace. 
Når der skal skrives til en af PSoC's vil dette gøres i driverens I2C\_interrupt\_write funktion. Her vil information fra user space kopieres til kernelspace, hvorefter funktionen I2C\_master\_send bliver brugt til at sende til den givne klient(her bruges I2C\_Client structet for enheden).
For en mere uddybende beskrivelse af driveren ses afsnit \fullref{sec:I2C_interruptDriver_design_bilag} i bilag.
\end{document}