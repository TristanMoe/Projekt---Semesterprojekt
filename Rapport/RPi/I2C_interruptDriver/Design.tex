\documentclass[Rapport/RPi/RPI.tex]{subfiles}
\begin{document}
\subsubsection{Softwaredesign}\label{sec:I2C_interruptDriver_design}
I dette afsnit beskrives software design for I2C\_interruptDriver. Driveren er lavet, som en char driver. Driveren har til formål at skrive til og læse fra de forskellige PSoC's i systemet ved hjælp af I2C. Der laves til dette formål en platform driver. For at være i stand til at bruge I2C i vores driver har vi gjort brug af biblioteket i2c.h.\\

Dette bibliotek bruges til at få adgang til en I2C bus ved hjælp af et i2c\_board\_info struct, der laves for alle PSoC enheder. Dette struct indeholder information om adresse og navn på PSoC enhederne, der nu vil refereres til som klienter. For alle klienter laves et struct (i2c\_interrupt\_device) med alle nødvendige informationer for klienterne, såsom navn og minor nummer mm.

Da denne driver er en platform driver, skal der tilføjes en probe og remove funktion. Driveren opretter noder for alle klienterne i probe.
I probe vil alt initiering af alle I2C klienterne finde sted, samt initiering af gpio's brugt til interrupts. Information omkring gpio'er brugt til interrupts er defineret i et device tree overlay. I probe vil disse informationer trækkes ud. I driverens open funktion vil interrupts initieres og ligeledes frigives i release funktionen. \\

Driverens read funktion skal læse fra alle PSoC's (en for hver node). Når i2c\_interrupt\_read kaldes, vil den med det samme lægge sig til at sove. Dette gør den for efterfølgende at skulle vækkes af et interrupt fra en af klienterne. Når den vækkes, vil information fra klienterne læses ved hjælp af funktionen i2c\_master\_recv(funktion i i2c.h). Informationen læst vil herefter kopieres til userspace. \\

Når der skal skrives til en af klienterne, gøres dette via driverens i2c\_interrupt\_write funktion. Her vil information fra user space kopieres til kernelspace, hvorefter funktionen i2c\_master\_send(funktion i i2c.h) bliver brugt til at sende til det givne klienter (her bruges i2c\_Client structet for enheden).
For en mere uddybende beskrivelse af driveren ses afsnit \fullref{swdesign:sec:I2C_interruptDriver_design_bilag} i bilaget \textbf{Softwaredesign}.
\end{document}