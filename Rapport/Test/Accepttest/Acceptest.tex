\documentclass[Rapport/Rapport_main.tex]{subfiles}
\begin{document}
\section{Accepttest}
Accepttesten for både de funktionelle og ikke-funktionelle krav blev specificeret i forhold til kravspecifikationen set i bilag "Kravspecifikation". 

Accepttesten for både de funktionelle og ikke-funktionelle krav blev foretaget d. 14-12-2018. Resultaterne er opsummeret i tabellerne \ref{tab:Accepttest_funktionelle_resultater} og \ref{tab:Accepttest_ikke_funktionelle_resultater}, som er vist nedenfor.
Der var flere punkter som fejlede, da vi som gruppe havde valgt at nedprioritere visse fysiske parametre for produktet. 

% Godkendt, delvist godkendt, ikke godkendt
\begin{table}[H]
\centering
\begin{tabular}{|L{0.3\textwidth}|L{0.3\textwidth}|L{0.3\textwidth}|}
\hline
\textbf{Testgrundlag} & \textbf{Status} \\ \hline
Use Case 1 & \\ \hline
Use Case 2 & \\ \hline
Use Case 3 & \\ \hline
Use Case 4 & \\ \hline
\end{tabular}
\caption{Accepttest resultat funktionelle krav}
\label{tab:Accepttest_funktionelle_resultater}
\end{table}

\begin{table}[H]
\centering
\begin{tabular}{|L{0.3\textwidth}|L{0.3\textwidth}|L{0.3\textwidth}|}
\hline
\textbf{Testgrundlag} & \textbf{Status} \\ \hline
Fysiske parametre & \\ \hline
User Interface & \\ \hline
Ydeevne & \\ \hline
Pålidelighed & \\ \hline
\end{tabular}
\caption{Accepttest resultat ikke-funktionelle krav}
\label{tab:Accepttest_ikke_funktionelle_resultater}
\end{table}

\textbf{Funktionelle krav:}\\
\textbf{UC1:}\\
\textbf{UC2:}\\
\textbf{UC3:}\\
\textbf{UC4:}\\

\textbf{Ikke-funktionelle krav:}\\
\textbf{Fysiske parametre:} \\
\textbf{User Interface:} \\ 
\textbf{Ydeevne:} \\
\textbf{Pålidelighed:} \\






\end{document}