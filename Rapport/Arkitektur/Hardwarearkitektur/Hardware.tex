\documentclass[Rapport/Rapport_main.tex]{subfiles}
\begin{document}
\subsection{Hardwarearkitektur}
Der beskrives i de følgende afsnit, hardwarearkitekturen for Beer Pong bordet. Hardwarearkitekturen illustreres ved brug af Block Definition Diagrams(BDD), hvor der hertil laves en beskrivelse af blokkene. Derefter laves et Internt Block Diagram (IBD) til at vise grænseflader og interne forbindelser i blokken. I de følgende diagrammer vil der være blokke markeret med \textbf{grøn} og \textbf{orange}. De \textbf{grønne} blokke er indkøbte blokke, da udviklingen af disse komponenter ikke prioriteres i projektet. De \textbf{orange} blokke er også indkøbte komponenter, men her er komponenterne modificeret med egen software.
\subsubsection{Overordnet Hardware Arkitektur}
Den overordnede hardwarearkitektur kan ses i et BDD for Beer Pong bordet i figur \ref{fig:rap_overall_hardware_bdd}.

\begin{figure}[H]
    \centering
    \makebox[\textwidth][c]{%
        \includegraphics[width=1.3\columnwidth,trim={0.24in 0.24in 0.24in 0.24in},clip, page=2]{Arkitektur/graphics/BDD_og_IBD.pdf}
    }
    \caption{Overordnet blok definitionsdiagram for systemet.}
    \label{fig:rap_overall_hardware_bdd}
\end{figure}
Tilsvarende laves der et IBD for det overordnede system, hvilket ses i figur \ref{fig:rap_overall_hardware_ibd}.

\begin{figure}[H]
    \centering
    \makebox[\textwidth][c]{%
        \includegraphics[width=1.3\columnwidth,trim={0.24in 0.24in 0.24in 0.24in},clip, page=5]{Arkitektur/graphics/BDD_og_IBD.pdf}
    }
    \caption{Overordnet intern blokdiagram for systemet.}
    \label{fig:rap_overall_hardware_ibd}
\end{figure}
IBD'et i figur \ref{fig:rap_overall_hardware_ibd} viser som sagt de interne forbindelser i systemet. Det vigtige at observere på IBD'et er kommunikationen mellem RPi (master) og de to Player sides og Ball dispenser (slaver). Her bruges der som beskrevet i analysen \fullref{analyse:sec:comm_analyse} I2C. En ulempe ved I2C er, at det altid er masteren, som skal initiere kommunikationen. Det vil kræve, at den hele tiden skal polle oplysninger fra slaverne for fx. at få oplysninger, hvornår der indsættes en mønt. Dette anses som værende upraktisk. Derfor tilføjes der for hver slave en interruptlinje,  som slaven trækker lav, når den har information, den vil videregive til masteren/RPi. Disse interruptslinjer er ligesom signalerne for I2C protokollen open-drain. Dette gøres for, at der kan håndteres kommunikation mellem et 3.3V system og et 5V system.\\
Nu hvor systemet er beskrevet overordnet, vil der blive gået mere i dybden med de enkelte blokke.
\subsubsection{Playerside - Hardwarearkitektur}
Til at starte med kigges der på hardwarearkitekturen for Playerside. Til at få overblik over de blokke som Playerside består af, laves et BDD, der kan ses på figur \ref{fig:rap_playerside_hardware_bdd}. Den fulde hardwarearkitektur for Playerside kan ses i \textbf{Arkitektur} dokumentet i afsnit \fullref{arch:sec:playerside_hardware}.

\begin{figure}[H]
    \centering
    \makebox[\textwidth][c]{%
        \includegraphics[width=1.3\columnwidth,trim={0.24in 0.24in 0.24in 0.24in},clip, page=3]{Arkitektur/graphics/BDD_og_IBD.pdf}
    }
    \caption{Blok definitionsdiagram for Player side.}
    \label{fig:rap_playerside_hardware_bdd}
\end{figure}
\textbf{Playerside} består af 6 \textbf{Cup Holders}, der hver især består af \textbf{Cup light} og \textbf{Cup sensor}. Cup Holder sørger altså for lyset og detektering af \textit{Game Events}.\\ 
Der skal også være en \textbf{Cup Holder Controller}, som styrer lyset på samtlige \textbf{Cup Holders} og samler og filtrerer sensorsignalet fra samtlige \textbf{Cup Holders}. \\ 
Til sidst er der indkøbt en PSoC Player Side, som er en PSoC 5LP Dev kit\cite{psoc5lp}, hvor der er implementeret softwaren \textbf{PlayerSideApp} på. Derfor er blokken orange. Denne blok behandler sensor outputtet fra \textbf{Cup Holder Controller} og styrer lyset gennem \textbf{Cup Holder Controller}.\\
De interne signaler for Playerside kan ses i IBD'et i figur \ref{fig:rap_playerside_hardware_ibd}.
\begin{figure}[H]
    \centering
    \makebox[\textwidth][c]{%
        \includegraphics[width=1.3\columnwidth,trim={0.24in 0.24in 0.24in 0.24in},clip, page=6]{Arkitektur/graphics/BDD_og_IBD.pdf}
    }
    \caption{Intern blokdiagram for Player side.}
    \label{fig:rap_playerside_hardware_ibd}
\end{figure}


\subsubsection{Ball dispenser - Hardwarearkitektur}
Det næste der kigges på er hardwarearkitekturen for Ball dispenser. Denne blok findes der kun én af i systemet, og BDD'et for delsystemet ses i figur \ref{fig:rap_balldispenser_hardware_bdd}. Den fulde hardwarearkitektur for Ball dispenser kan ses i \textbf{Arkitektur} dokumentet i afsnit \fullref{arch:sec:balldispenser_hardware}.
\begin{figure}[H]
    \centering
    \makebox[\textwidth][c]{%
        \includegraphics[width=1.3\columnwidth,trim={0.24in 0.24in 0.24in 0.24in},clip, page=4]{Arkitektur/graphics/BDD_og_IBD.pdf}
    }
    \caption{Blok definitionsdiagram for Ball dispenser.}
    \label{fig:rap_balldispenser_hardware_bdd}
\end{figure}
Det ses af figur \ref{fig:rap_balldispenser_hardware_bdd}, at Ball dispenser består af en \textbf{Coin Collector}, hvor der indkastes en mønt, der detekteres af en \textbf{Coin Sensor} og behandles af en \textbf{Motor}. Hvis den rigtige mønt er indkastet, så står \textbf{Ball release} for at dispensere bolde, hvilket gøres med en  \textbf{Motor}. Til begge \textbf{Motors} hører en \textbf{Motor controller} til at styre motoren. Ball dispenser skal opdatere \textbf{Status LEDs} i forhold til om, der mangler bolde, eller dispenseren er fuld, hvilket detekteres med \textbf{Ball count sensor}. Af figuren ses det også, at \textbf{Power Supply}, \textbf{Motor}, \textbf{Motor controller} og \textbf{PSoC ball dispenser} er indkøbt, og \textbf{PSoC ball dispenser} anvender egen software, \textbf{BallDispenserApp}.\\
De interne signaler i blokken kan desuden ses i figur \ref{fig:rap_balldispensere_hardware_ibd}.

\begin{figure}[H]
    \centering
    \makebox[\textwidth][c]{%
        \includegraphics[width=1.3\columnwidth,trim={0.24in 0.24in 0.24in 0.24in},clip, page=7]{Arkitektur/graphics/BDD_og_IBD.pdf}
    }
    \caption{Intern blokdiagram for Ball dispenser.}
    \label{fig:rap_balldispensere_hardware_ibd}
\end{figure}


\subsubsection{RPi - Hardwarearkitektur}
RPi er hjernen i systemet, der er derfor ikke mange hardwareblokke, der hører til her, da det meste logik er lavet i software. BDD'et for RPi kan ses i figur \ref{fig:rap_rpi_hardware_bdd}. Hardwarearkitekturen for RPI kan ses i afsnit \fullref{arch:sec:RPi_hardware_ports} af i \textbf{Arkitektur} dokumentet. 
\begin{figure}[H]
    \centering
    \makebox[\textwidth][c]{%
        \includegraphics[width=1\columnwidth,trim={0.24in 0.24in 0.24in 0.24in},clip, page=9]{Arkitektur/graphics/BDD_og_IBD.pdf}
    }
    \caption{Blok definitionsdiagram for RPi.}
    \label{fig:rap_rpi_hardware_bdd}
\end{figure}
Figur \ref{fig:rap_rpi_hardware_bdd} viser at \textbf{RPi Power Supply} og \textbf{RPi Zero W} er indkøbt, hvor der på RPi'en er softwaren \textbf{RPiApp}. Den eneste egentlige hardwareblok er \textbf{Pull up resistors}, der er de modstande, der anvendes til interrupts fra de andre delsystemer. Det er vigtigt, at det er pull-up modstande til 3.3V og ikke 5V. Det er ikke nødvendigt med pull-up modstande til I2C signalerne, da RPi Zero W allerede har interne pull-up modstande \autocite{RPiPins}  \\
De interne signaler for denne blok kan ses i figur \ref{fig:rap_rpi_hardware_ibd}.
\begin{figure}[H]
    \centering
    \makebox[\textwidth][c]{%
        \includegraphics[width=1.3\columnwidth,trim={0.24in 0.24in 0.24in 0.24in},clip, page=10]{Arkitektur/graphics/BDD_og_IBD.pdf}
    }
    \caption{Intern blokdiagram for RPi.}
    \label{fig:rap_rpi_hardware_ibd}
\end{figure}



\end{document}