\documentclass[Rapport/Rapport_main.tex]{subfiles}
\begin{document}
\section{Metode og proces}
I dette afsnit beskrives processen af et 3. semester projekt omhandlende udviklingen af et interaktivt Beer Pong bord. Derfor er dette afsnit en opsummering af bilaget \textbf{Proces}. Udover processen vil der også være en beskrivelse af de metoder og redskaber, der er blevet anvendt under forløbet. \\
Som det første er det værd at kigge på de krav der er for processen i forbindelse med 3. semester projektet\cite{Universitet2018}. De to krav, der har med processen at gøre er:
\begin{itemize}
    \item "Anvendelsen af processer og metoder kendt fra projektet på 2. semester, men anvendt iterativt over alle faser."
    \item "Iterativ arbejdsmetode, SCRUM, orienteret mod at udvikle nye produkter baseret på HW og SW."
\end{itemize}
Det er helt tydeligt her at dette projekt skal udvikles iterativ gennem Scrum, og derfor vil det næste underafsnit fokusere på det.

\subsection{Andvendelse af Scrum i PRJ3, Gruppe7}
Som sagt er dette afsnit dedikeret til at beskrive den anvendelse af Scrum, der har været for Gruppe7. Gennem den proces der har været i forbindelse med projektet, er der blevet gjort mange forskellige erfaringer i forhold til, hvad der har fungeret for gruppen. Til at starte med er der udarbejdet en overordnede skitse, der er blevet lavet til at beskrive anvendelsen Scrum i PRJ3 Gruppe7. Denne kan ses i figur \ref{fig:rap_scrum_usage}.
\begin{figure}[H]
    \centering
    \includegraphics[width=\textwidth]{Processdokument/graphics/Scrum_usage.png}
    \caption{Skitse for hvordan der er anvendt SCRUM i PRJ3 Gruppe7}
    \label{fig:rap_scrum_usage}
\end{figure}
Af figur \ref{fig:rap_scrum_usage} ses det \textbf{i øverste venstre hjørne}, hvordan gruppen selv har varetaget rollen som Product Owner, da projektet der udvikles er specificeret af dem og de bestemmer hvilke funktioner og egenskaber, der skal prioriteres i deres projekt. Samme sted ses det også, hvordan rollen som Scrum Master er gået på tur efter hvert sprint, da alle skulle prøve at have ansvaret og erfaringerne, der følger med rollen.\\
\textbf{I midten i venstre side} ses det, hvordan der til Sprint-planlægningsmøder er blevet udvalgt de relevante tasks fra backloggen, der skulle med i det nye sprint. Hertil laves en sprint backlog, der indeholder alle tasks i sprintet. Der var desuden gode erfaringer med at sætte konkrete målbare mål i sprintplanlægningen. Det kunne for eksempel være at have demoer klar til vejleder møde eller dokumenter færdiggjort.\\
Efter at sprintet er planlagt, kan det ses \textbf{i midten}, at der udføres et sprint med en varighed af 2 uger, hvor der undervejs i sprintet laves Standup Meetings, på de aftalte dage. Til disse møder opdateres hele teamet i forhold til, hvor langt de enkelte er med deres opgaver, og om der er problemer eller mangler hjælp et sted.\\
Til slut i sprintet, ses det \textbf{i højre side}, at der så laves et møde med vejleder, hvor der evalueres Risikoer i forbindelse med projektet. Ud fra disse risikoer kan der planlægges tasks. Desuden stilles der forskellige spørgsmål i forbindelse med usikkerheder eller problemer i projektet. Efter et sprint er der selvfølgelig også færdiggjort arbejde, det kan være dokumentation, design eller implementation. Undervejs i forløbet sendes det til et eksternt review hos en anden 3. semesterprojekt gruppe. Til sidst laves et retrospektiv på Sprintet, hvor der bliver nedskrevet positive og negative ting fra det pågældende sprint. Der opsættes så to-tre af de negative punkter, der fokuseres på at forbedre i det næste sprint. Aktionspunkterne fra forrige Sprint blev også taget op for at se om, der rent faktisk var sket en forbedring.\\\\
Til at gøre Scrum nemmere at facillitere er der blevet anvendt Redmine som værktøj. Redmine indeholder netop de relevante redskaber i forhold til Sprint board, oversigt over tasks, backlogs og så videre. Desuden har Redmine en Wiki funktion, der blev anvendt af Scrum master til at nedskrive information omkring planlægning og retrospektiv.\\
Hvis der er interesse for anvendelsen af Scrum, samt de erfaringer, der er gjort i løbet af sprintet, så henvises der til afsnit \fullref{proces:sec:experiences_w_scrum} i Proces-dokumentet.

\subsection{Iterativ ASE-udviklingsmodel og SYSML}
Som krav for 3. semester projektet var der, at der skulle anvendes metoder og processer, som blev anvendt til 2. semester projektet. På 2. semester blev der til udvikling anvendt ASE-udviklingsmodellen, der kan ses på figur \ref{fig:rap_ase_model}.
\begin{figure}[H]
    \centering
    \includegraphics[width=\textwidth]{Processdokument/graphics/ASE_model.png}
    \caption{Udviklingsmodel fra ASE \cite{vejledning_prj3}}
    \label{fig:rap_ase_model}
\end{figure}
ASE-modellen i figur \ref{fig:rap_ase_model}, blev anvendt iterativt henover alle dele. Det vil altså siges at hardware og software blev designet og implementeret iterativt, men også at mange af dokumenterne blev udarbejdet iterativt. Det gjorde de idet de i løbet af sprintene hele tiden blev forbedret og rettet.\\\\

Derudover blev der også anvendt SysML, som redskab til at modelere og beskrive systemet. I SysML blev der anvendt de begreber og metoder, der er stiftet bekendskab med i 2. semester kurset Indledende System Engineering (ISE). De anvendte SysML metoder kan ses i figur \ref{fig:rap_sysml_usage}.
\begin{figure}[H]
    \centering
    \includegraphics[width=\textwidth]{Processdokument/graphics/Sysml_usage.png}
    \caption{Udarbejdet diagram over de anvendte SysML i Gruppe7}
    \label{fig:rap_sysml_usage}
\end{figure}
Af figur \ref{fig:rap_sysml_usage} ses det at opførelsen af systemet beskrives gennem Sekvens-, Statemachine- og Use Case-diagrammer.
Strukturen af systemet beskrives ved Block Definition- og Internal Block-diagram. Her er Class-Diagram inkluderet fra UML, da den beskriver strukturen og relationerne i software.\\
I modsætning til sidste semester, hvor diagrammerne var fastlagt ved starten af udviklingsfasen, så blev SysML anvendt iterativt ved at diagrammer blev rettet, ændret og forbedret undervejs. 

\end{document}