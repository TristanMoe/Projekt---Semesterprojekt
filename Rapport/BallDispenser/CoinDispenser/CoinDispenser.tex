\documentclass[Rapport/Rapport_main.tex]{subfiles}
\begin{document}
\subsection{CoinDispenser}
\subsubsection{Hardwaredesign}
CoinDispenseren trengte en løsning på hvordan den kunne sortere og akseptere/avslå mynter som kom inn i systemet. Dette blir gjort med en statisk myntsorterings maskin som automatisk avslår alle mynter som ikke er 5kr. Den er konstruert som en sklie som myntene sklir ned på med 2 huller, ett som er 27.5mm som er stort nåkk til at alle mynter som ikke er 5kr faller gjennom og ett hvor 5 kr myntene faller gjennom.

\begin{figure}[H]
    \centering
    \includegraphics[width=\linewidth]{Rapport/BallDispenser/CoinDispenser/graphics/coinmaster.png}
    \caption{CoinDispenser CAD Model created in Autodesk Inventor}
    \label{fig:CoinCAD}
\end{figure}

Myntene er detektert ved av to metall plater som er nær hverandre, nå mynten faller ned på platen så kortsluttes koblingen og det sendes et interrupt til PSoCen.

\begin{figure}[H]
    \centering
    \includegraphics[width=0.3\linewidth]{Rapport/BallDispenser/CoinDispenser/graphics/CoinDispInterrupt.png}
    \caption{Coin Detection with short circuit setup}
    \label{fig:coin_disable_interrupt_diagram}
\end{figure}

De sorterte myntene er akseptert eller avslått ved hjelp av en stepper motor koblet til en plate med et hull som ligger over metallplaten som er beskrevet over. Denne motoren er kontrollert av PSoCen og dreier med eller mot klokken avhengig av om den skal akseptere eller avslå myntene. Dokumentasjon av motor kontrolleren kan finnes i \textbf{Softwaredesign} dokumentet, avsnitt \fullref{swdesign:subsec:motorControlDesign}

\begin{figure}[H]
    \centering
    \includegraphics[width=1\textwidth]{Rapport/BallDispenser/CoinDispenser/graphics/coinmaster-under.png}
    \caption{CoinDispenser underside with steppermotor connected}
    \label{fig:coin_dispenser_bottom}
\end{figure}

\subsubsection{Softwaredesign}
\begin{figure}[H]
    \centering
    \includegraphics[width=1\textwidth]{Rapport/BallDispenser/CoinDispenser/graphics/CoinSTM.png}
    \caption{CoinDispenser State Machine}
    \label{fig:CoinSTM}
\end{figure}
I dette avsnitt beskrives designet for CoinDispenser classen på balldispenseren. Denne klassen skal kontrollere hardwaren til CoinDispenseren. Modulet bruker en external interrupt til å detektere mynter som kommer inn i systemet. Når en mynt detekteres kalles funksjonen HandleCoin som kan sees i arkitekturen, denne kaller funksjonene acceptCoin /rejectCoin etter som hvilket stadium CoinDispenser er i. Hvis systemet er i i DispenserEnabled så aksepteres mynten og det sendes beskjed til RPi om at en mynt er akseptert via Contolleren. Hvis systemet er i DispenserDisabled så avvises mynten og systemet fortsetter å vente.
\subsubsection{Implementering}
Implementeringen av CoinDispenser er med bruk av PSoC Creator som er utviklings plattformen til PSoC. Dokumentasjon kan sees i \textbf{Softwaredesign} dokumentet, avsnitt \fullref{swdesign:sec:CoinDisp}
\subsubsection{Modultest}
Modul testen til CoinDispenser ble gjennomført som en blackbox test med et unntak hvor vi må endre state fra enabled til disabled. Testen ble gjennomført ved a skli mynter ned sorterings sklien hvor de så lander på mynt deteksjons planten og observere om de blir detektert og håntert riktig i henhold til kravene. For tester av de individuelle komponenter og tabeller av resultatene refereres det til \textbf{Modul Test} dokumentet, avsnitt \fullref{modultest:sec:CoinModulTest}\\
\end{document}