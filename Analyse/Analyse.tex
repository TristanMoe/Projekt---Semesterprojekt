\documentclass[a4paper,12pt,fleqn,oneside]{article} 
%\documentclass[a4paper,12pt,fleqn,oneside]{article} 	% Openright aabner kapitler paa hoejresider (openany begge)

%===== Projekt konstanter =====
\newcommand{\kursusTitel}{Semesterprojekt 3}
\newcommand{\linje}{E/IKT}
\newcommand{\semester}{3}
\newcommand{\system}{Beer Pong Table}
\newcommand{\rapportType}{Proces}
\newcommand{\gruppeNr}{7}
\newcommand{\vejleder}{Martin Ansbjerg Kjaer}
\newcommand{\afleveringsdato}{19. december 2018}

%%%% PAKKER %%%%
% ¤¤ Oversaettelse og tegnsaetning ¤¤ %
\usepackage[utf8]{inputenc}					% Input-indkodning af tegnsaet (UTF8)
\usepackage[english, danish]{babel}			% Dokumentets sprog
\usepackage[T1]{fontenc}					% Output-indkodning af tegnsaet (T1)
\usepackage{ragged2e,anyfontsize}			% Justering af elementer

% ¤¤ Figurer og tabeller (floats) ¤¤ %
\usepackage{wrapfig}                        % text wrapping
\usepackage{graphicx} 						% Haandtering af eksterne billeder (JPG, PNG, PDF)
\usepackage{caption}
\usepackage{subcaption}
\usepackage{multirow}
% Fletning af raekker og kolonner (\multicolumn og \multirow)
\usepackage{makecell}                       % Line breaks i tabelceller med \makecell{bla bla \\ bla bla}
\usepackage{colortbl} 						% Farver i tabeller (fx \columncolor, \rowcolor og \cellcolor)
\usepackage[dvipsnames,table,longtable,x11names]{xcolor}

%Tabel automatisk ny linje
\usepackage{array}
\newcolumntype{L}[1]{>{\raggedright\let\newline\\\arraybackslash\hspace{0pt}}m{#1}}
\newcolumntype{C}[1]{>{\centering\let\newline\\\arraybackslash\hspace{0pt}}m{#1}}
\newcolumntype{R}[1]{>{\raggedleft\let\newline\\\arraybackslash\hspace{0pt}}m{#1}}

% Definer farver med \definecolor. Se mere: http://en.wikibooks.org/wiki/LaTeX/Colors
\usepackage{flafter}						% Soerger for at floats ikke optraeder i teksten foer deres reference
\let\newfloat\relax 						% Justering mellem float-pakken og memoir
\usepackage{float}							% Muliggoer eksakt placering af floats, f.eks.
\usepackage{afterpage}
%\usepackage{scrextend}                      % labeling lister
\usepackage{chngcntr} % kontroller nummerering af floats
\counterwithin{figure}{section} % sæt nummerering efter sektion
\counterwithin{table}{section} % sæt nummerering efter sektion

% ¤¤ Matematik mm. ¤¤
\usepackage{amsmath,amssymb,stmaryrd} 		% Avancerede matematik-udvidelser
\usepackage{mathtools}						% Andre matematik- og tegnudvidelser
\usepackage{textcomp}                 		% Symbol-udvidelser (f.eks. promille-tegn med \textperthousand )
\usepackage{siunitx}						% Flot og konsistent praesentation af tal og enheder med \si{enhed} og \SI{tal}{enhed}
\sisetup{output-decimal-marker = {,}}		% Opsaetning af \SI (DE for komma som decimalseparator)

% ¤¤ Misc. ¤¤ %
\usepackage{listings}						% Placer kildekode i dokumentet med \begin{lstlisting}...\end{lstlisting}

%Custom C Code Style
\lstdefinestyle{customc}{
    breaklines=true,
    language=C,
    basicstyle=\small\sffamily,
    numbers=left,
    numberstyle=\tiny\color{gray},
    frame=tb,
    columns=fullflexible,
    showstringspaces=false,
    keywordstyle=\bfseries\color{green!40!black},
    commentstyle=\itshape\color{purple!40!black},
    identifierstyle=\color{blue},
    stringstyle=\color{orange},
}


\usepackage{blindtext}
\usepackage{lipsum}							% Dummy text \lipsum[..]
\usepackage[shortlabels]{enumitem}			% Muliggoer enkelt konfiguration af lister
\usepackage{pdfpages}						% Goer det muligt at inkludere pdf-dokumenter med kommandoen \includepdf[pages={x-y}]{fil.pdf}
\usepackage[bottom]{footmisc}               % Saetter footnotes i bunden af siden
\pdfoptionpdfminorversion=6					% Muliggoer inkludering af pdf dokumenter, af version 1.6 og hoejere
\pretolerance=2500 							% Justering af afstand mellem ord (hoejt tal, mindre orddeling og mere luft mellem ord)


%%%% BRUGERDEFINEREDE INDSTILLINGER %%%%

\linespread{1,1}							% Linie afstand

% ¤¤ Visuelle referencer ¤¤ %
\usepackage[colorlinks,pdfencoding=auto]{hyperref}			% Danner klikbare referencer (hyperlinks) i dokumentet.
\hypersetup{colorlinks = true,				% Opsaetning af farvede hyperlinks (interne links, citeringer og URL)
    linkcolor = black,
    citecolor = black,
    urlcolor = black
}

%%%% TODO-NOTER %%%%
\usepackage[danish, colorinlistoftodos]{todonotes}
%\usepackage[colorinlistoftodos]{todonotes}

%%%% TABEL BAGGRUNDSFARVER %%%%
\definecolor{aublueclassic}{RGB}{0,61,115}
\definecolor{aubluedark}{RGB}{0,37,70}
\definecolor{aucyan}{RGB}{225,248,253}
%\definecolor{aucyan}{RGB}{55,160,203}
\definecolor{aucyandark}{RGB}{0,62,92}
\definecolor{lightGray}{RGB}{153,153,153}
\definecolor{darkGray}{RGB}{119,119,119}
\definecolor{khaki}{RGB}{240,230,140}
\definecolor{lavender}{RGB}{230,230,250}

%indentering af section
\usepackage{changepage}

%Code highlighting
\usepackage{minted}

%%%% Tabs %%%%
\usepackage{tabto}
\NumTabs{10}

%%%% REFERENCE TIL SECTION-NAME %%%%
\usepackage{nameref}
\newcommand*{\fullref}[1]{\textbf{\ref{#1} \nameref{#1}}} % One single link

%sidehoved
\usepackage{fancyhdr}
\usepackage{lastpage}

%referer til afsnit i andre dokumenter
\usepackage{xr}

\externaldocument[arch:]{aux_files/Arkitektur}
\externaldocument[accepttestspec:]{aux_files/Accepttestspecifikation}
\externaldocument[hwdesign:]{aux_files/HardwareDesign}
\externaldocument[analyse:]{aux_files/Analyse}
\externaldocument[integration:]{aux_files/Integrationstest}
\externaldocument[kravspec:]{aux_files/Kravspecifikation}
\externaldocument[modultest:]{aux_files/Modultest}
\externaldocument[proces:]{aux_files/Procesdokument}
\externaldocument[swdesign:]{aux_files/Softwaredesign}

%%%% biblatex %%%%
\usepackage[backend=bibtex,style=ieee]{biblatex}
\bibliography{Litteratur} 

%%%% KRAV numerering %%%%

\newcounter{req} \setcounter{req}{0}
\newcounter{subreq}[req] \setcounter{subreq}{0}
 
\renewcommand{\thereq}{\arabic{req}}
\renewcommand{\thesubreq}{\arabic{req}.\arabic{subreq}}
 
\newenvironment{req}[1]%
{
\refstepcounter{req}{K\thereq}}{}
\newenvironment{subreq}[1]%
{
\refstepcounter{subreq}{\noindent K\thesubreq}}{}


%%%% paragraph %%%%
%\usepackage{titlesec}
%\setcounter{secnumdepth}{4}

%\titleformat{\paragraph}
%{\normalfont\normalsize\bfseries}{\theparagraph}{1em}{}
%\titlespacing*{\paragraph}
%{0pt}{3.25ex plus 1ex minus .2ex}{1.5ex plus .2ex}

% ========== PAKKER DER SKAL LOADES TIL SIDST ==================
%\usepackage{xcolor}
%\usepackage{listings}
\usepackage{csquotes}           %så holder bilatex kæft
\usepackage{subfiles}


%%EDS SHIT
\PassOptionsToPackage{unicode=true}{hyperref} % options for packages loaded elsewhere
\PassOptionsToPackage{hyphens}{url}
%
\usepackage{lmodern}
\usepackage{amssymb,amsmath}
\usepackage{ifxetex,ifluatex}
\ifnum 0\ifxetex 1\fi\ifluatex 1\fi=0 % if pdftex
  \usepackage[T1]{fontenc}
  \usepackage[utf8]{inputenc}
  \usepackage{textcomp} % provides euro and other symbols
\else % if luatex or xelatex
  \usepackage{unicode-math}
  \defaultfontfeatures{Ligatures=TeX,Scale=MatchLowercase}
\fi
% use upquote if available, for straight quotes in verbatim environments
\IfFileExists{upquote.sty}{\usepackage{upquote}}{}
% use microtype if available
\IfFileExists{microtype.sty}{%
\usepackage[]{microtype}
\UseMicrotypeSet[protrusion]{basicmath} % disable protrusion for tt fonts
}{}
\IfFileExists{parskip.sty}{%
\usepackage{parskip}
}{% else
\setlength{\parindent}{0pt}
\setlength{\parskip}{6pt plus 2pt minus 1pt}
}
\usepackage{hyperref}
\hypersetup{
            pdfborder={0 0 0},
            breaklinks=true}
\urlstyle{same}  % don't use monospace font for urls
\usepackage{longtable,booktabs}
% Fix footnotes in tables (requires footnote package)
\IfFileExists{footnote.sty}{\usepackage{footnote}\makesavenoteenv{longtable}}{}
\setlength{\emergencystretch}{3em}  % prevent overfull lines
\providecommand{\tightlist}{%
  \setlength{\itemsep}{0pt}\setlength{\parskip}{0pt}}
% Redefines (sub)paragraphs to behave more like sections
\ifx\paragraph\undefined\else
\let\oldparagraph\paragraph
\renewcommand{\paragraph}[1]{\oldparagraph{#1}\mbox{}}
\fi
\ifx\subparagraph\undefined\else
\let\oldsubparagraph\subparagraph
\renewcommand{\subparagraph}[1]{\oldsubparagraph{#1}\mbox{}}
\fi

% set default figure placement to htbp
\makeatletter
\def\fps@figure{htbp}
\makeatother


\def\titlename{Analyse}



\begin{document}
%===============FORSIDE======================
\pagestyle{fancy}
\fancyhf{}
\lhead{\kursusTitel\\Gruppe \gruppeNr}
\rhead{\titlename\\ \nouppercase{\leftmark}}
\lfoot{\afleveringsdato\\}
\rfoot{Side \thepage\space af \pageref{LastPage}}

\begin{titlepage}

\newcommand{\HRule}{\rule{\linewidth}{0.5mm}} % Defines a new command for the horizontal lines, change thickness here

\center % Center everything on the page
 
%----------------------------------------------------------------------------------------
%	HEADING SECTIONS
%----------------------------------------------------------------------------------------

\textsc{\LARGE Aarhus Universitet }\\[0.3cm] % Name of your university/college
\textsc{\Large \kursusTitel }\\[0.3cm]
\textsc{\Large Gruppe \gruppeNr }\\[0.5cm] % Major heading such as course name
 % Minor heading such as course title

%----------------------------------------------------------------------------------------
%	TITLE SECTION
%----------------------------------------------------------------------------------------

\HRule \\[0.4cm]
{ \huge \bfseries \titlename}\\[0.03cm]{\system} % Title of your document
\HRule \\[1.5cm]

 
%----------------------------------------------------------------------------------------
%	AUTHOR SECTION
%----------------------------------------------------------------------------------------

\begin{minipage}{0.4\textwidth}
\begin{flushleft} \small
\emph{Gruppemedlemmer:} 
\\Aaron Becerril Sanchez\\ (201709281)
\\Edward Hestnes Brunton\\ (AU576633)
\\Marcus Gasberg\\ (201709164) 
\\Martin Gildberg Jespersen\\ (201706221) 
\\Martin Lundberg\\ (201707003) 
\\Mathias Magnild Hansen\\ (201404884)
\\Nikolaj Holm Gylling\\ (201704806)
\\Tristan Moeller\\ (20176862)
\end{flushleft}
\end{minipage}
~
\begin{minipage}{0.4\textwidth}
\begin{flushright} \small
\emph{Vejleder:} \\
\vejleder \\ Civ. Ing., Ph.D., Adjunkt % Supervisor's Name
\end{flushright}
\end{minipage}\\[1cm]

% If you don't want a supervisor, uncomment the two lines below and remove the section above
%\Large \emph{Author:}\\
%John \textsc{Smith}\\[3cm] % Your name

%----------------------------------------------------------------------------------------
%	DATE SECTION
%----------------------------------------------------------------------------------------

{\large \afleveringsdato}\\[1cm] % Date, change the \today to a set date if you want to be precise

%----------------------------------------------------------------------------------------
%	LOGO SECTION
%----------------------------------------------------------------------------------------

\begin{figure}[H]
    \centering
    \includegraphics[width=0.15\textwidth]{Setup/graphics/AU.png}
\end{figure}

\vfill
\end{titlepage}
%\subfile{Analyse/Versionshistorik/Versionshistorik.tex}\newpage
\tableofcontents
\newpage
I dette dokument beskrives det analyse arbejde, hvor de forskellige overvejelser omkring løsninger kommer til udtryk. Der vil hertil lave en begrundelse for de grundlæggende valg af komponenter i systemet. \\\\
Kigges der på systemet i sit store hele, så kommer det til at bestå af to sider til holdene, hvor holdenes kopper kommer til at stå. Der kommer til at være en møntindkastning, hvor der indkastes en mønt for at starte spillet. Efter en mønt er indkastet, så skal der dispenseres bolde, der skal anvendes til spillet. Der skal altså derfor være noget noget til opbevaring og dispensering af bolde.\\ 
\subsection{System Opdeling}\label{sec:sys_opdeling}
Efter at have opstillet systemet i sit store hele, kan det være smart at opdele systemet.\\
Der kan f.eks. laves et centraliseret system, hvilket vil være den billige løsning, da der skal bruges et lavt antal af mikroprocessorer. Afvejningen her vil være at der skulle arbejdes på samme system, hvilket kan skabe problemer og forstyrrer udviklingen.\\
Delsystemer kan være opdelt ud fra deres ansvarsområder. Her kan laves 2 delsystemer af henholdsvis, hver side af Beer Pong-bordet. At dele siderne op for sig selv er praktisk, da hver side kun skal bekymre sig om det, der sker på sin egen side. Der kan også laves et delsystem af det, der skal tage imod en mønt og dispensere en bold. Der vælges et samlet delsystem for møntindkast og dispensering, da de to dele hænger sammen, da de fysiskset er placeret samme sted og ligesom på spillersiderne, så skal en detektering udløse en reaktion. Nu hvor der bestræbes efter høj samhørighed og lav kobling, så burde ingen af delsystemerne have ansvaret for at styre de andre og holde styr på de forskellige stadier i systemet. Derfor bør der også være et delsystem, der står for logikken og kontrollering af de andre delsystemer.\\
Da der er flere dele af systemet, der har unikke opgaver i forhold til de andre, så kan der her anvendes en separat CPU til hver del. Det vil altså sige, at der er en CPU til bolddispenseren, en CPU til hver spiller side og til sidst en CPU til styring af display og webserver. Dette valg sikrer at systemet er opdelt i relevante moduler, hvilket gør det nemmere at arbejde på, samt det gør det nemmere at udskifte eller lave ændringer på modulet uden det påvirker de andre. Det sikrer også den lave kobling mellem modulerne, da de bare skal videre kommunikere deres information til en controller, uden noget kendskab til de andre moduler.
\subsection{Spillersiderne som delsystemer}
Startes der med en analyse af det delsystemet omkring spillersiderne. Her skal der detekteres hvorvidt holdene fjerner, tilføjer eller rammer i en kop. Til dette skal der anvendes en form for sensor, der kan registrere alt dette. Når sådan et \textit{Game Event} registreres, så er det også vigtigt for en bruger at få visuel feedback på eventet. Denne feedback skal bestå af lys. Prisen for både sensor og lys spiller en væsentlig rolle, da der kommer til at være mange af dem, 6 steder på hver side af bordet. Det samme gør størrelsen på komponenterne, da kopperne skal stå rimelig tæt på hinanden.
Som CPU anvendes der en PSOC 5LP CY8C58LP \cite{psoc5lp}, der er et \textit{System On A Chip}, hvor der hører et udviklings IDE med, kaldet PSoC-creator. PSoC5 LP er smart til netop dette system, da der udvikles en prototype, og den er nem at eksperimentere og arbejde med, da den indeholder hardware komponenter på chippen(f.eks. timers, ADC osv). Desuden er det også et krav til systemet i forbindelse med 3. semesterprojektet \cite{Universitet2018}, at der anvendes en PSoC-platform.\\ PSoC-creator IDE'et gør det også nemt at udvikle på chippen, da der herigennem kan ændres på hardware på chippen og der dynamisk genereres API's til at anvende hardwaren. Derudover har IDE'et en gratis C-compiler, en editor og en debugger, der også kan anvendes på PSoC5 lp. Dette er især relevant, da der udvikles en prototype, men denne løsning er forholdsvis dyr. Hvis der derimod skulle laves et færdigt system, så skulle dette udviklingskit erstattes af en CPU i passende størrelse eller PSoC IC'en selv, samt de hardware komponenter, der blev anvendt på PSoC chippen.
\subsection{Bolddispenser som delsystem}
 Der skal styres hvor mange bolde der dispenseres, og da bolde skal flyttes fra systemet ud til brugeren, så må den have en aktuator. Derudover skal der detekteres, hvorvidt der mangler bolde eller om dispenseren er fuld, hvilket skal gøres med en sensor. \\
 En detektion af en mønter nødvendig for at starte spillet, hvilket igen kræver en sensor. Mønten skal også forkastes eller indtages i systemet, alt efter om det er den rigtige mønt, eller bolddispenseren er i låst tilstand. Dette kræver en aktuator.\\
Som CPU for denne del anvendes PSoC5 lp ud fra de samme argumenter, som ved spillersiderne.
\subsection{System Controller som delsystem}
Det sidste delsystem er hjernen i systemet. Den skal kunne tilknyttes andre CPU'er og kommunikere med dem gennem en udspecificeret kommunikationsprotokol. Den styrer stadierne i de andre delsystemer, samt flowet af spillet. Ud fra de funktionelle krav \fullref{kravspec:sec:funktionelle_krav} skal den også kunne hoste en WebServer og skrive status af systemet ud på et display.\\
Til dette anvendes der en RPi Zero W med indlejret Linux\cite{rpi_webpage}. Det er ideelt til netop denne opgave, da der kan laves drivere i forbindelse med kommunikation mellem processorene. Den har også HDMI til et display og  Wifi med de konfigurationer der skal til for, at den kan anvendes som en WebServer. Fra kravene 3. semesterprojekt\cite{Universitet2018} er der også givet, at der skal anvendes en indlejret linux-platform. 

\subsection{Kommunikation mellem delsystemer}\label{sec:comm_analyse}
Mellem alle delsystemer skal der specificeres en kommunikationsprotokol, så delsystemerne kan kommunikere internt. Til en start kan der laves et hierarki for systemet. Den øverste i hierarkiet må være hjernen af systemet \textsc{RPi}, og den kan derfor kaldes \textit{Master}. De andre delsystemer skal kun kommunikere med hjernen af systemet og kan derfor kaldes \textit{Slaves}. Nu hvor hierarkiet er specificeret kan der laves overvejelser omkring kommunikationsprotokollen.
\subsubsection{Kommunikationsprotokol}
På 3. semester som IKT- og E-studerende stifter man bekendtskab med kommunikationsprotokollerne I2C\cite{i2c_protocol} og SPI\cite{spi_protocol}. Af den grund omhandler overvejelsen disse 2. Som det første blev der opstillet fordele og ulemper for begge protokoller.\\
\textbf{SPI}
\begin{itemize}
    \item SPI er hurtigst, har en frekvens på 20MHz mod I2C's 5MHz på ultrafast mode.
\end{itemize}
\textbf{I2C}
\begin{itemize}
    \item I2C anvender færre linjer mellem Master og Slaves end SPI .
    \item I2C kan anvendes med både PSoC og RPi, uden modificering af PSoC (Da PSoC opererer ved 5V og RPi ved 3.3V). Det vil virke, fordi  I2C benytter open-drain og SPI ikke gør.
    \item Erfaringer siger at I2C er nemmest at implementere på PSoC. 
\end{itemize}
Det at I2C kan anvendes uden al for stor modificering af komponenterne, spiller en stor rolle, da denne kommunikationsprotokol kan anvendes med mindre risiko for beskadigelse af RPi, end hvis man tilsluttede en umodificeret PSoC til SPI.\\
En anden faktorder også snakker for I2C, er erfaringerne ved SPI på PSoC'en, der simpelthen giver for stor en risiko omkring kommunikationsprotokollen. Der vælges derfor \textbf{I2C}, som kommunikationsprotokol.




\end{document}
