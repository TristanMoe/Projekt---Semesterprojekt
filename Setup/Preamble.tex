%\documentclass[a4paper,12pt,fleqn,oneside]{article} 	% Openright aabner kapitler paa hoejresider (openany begge)

%===== Projekt konstanter =====
\newcommand{\kursusTitel}{Semesterprojekt 3}
\newcommand{\linje}{E/IKT}
\newcommand{\semester}{3}
\newcommand{\system}{Beer Pong Table}
\newcommand{\rapportType}{Proces}
\newcommand{\gruppeNr}{7}
\newcommand{\vejleder}{Martin Ansbjerg Kjær}
\newcommand{\afleveringsdato}{19. december 2018}

%%%% PAKKER %%%%
% ¤¤ Oversaettelse og tegnsaetning ¤¤ %
\usepackage[utf8]{inputenc}					% Input-indkodning af tegnsaet (UTF8)
\usepackage[english, danish]{babel}			% Dokumentets sprog
\usepackage[T1]{fontenc}					% Output-indkodning af tegnsaet (T1)
\usepackage{ragged2e,anyfontsize}			% Justering af elementer

% ¤¤ Figurer og tabeller (floats) ¤¤ %
\usepackage{wrapfig}                        % text wrapping
\usepackage{graphicx} 						% Haandtering af eksterne billeder (JPG, PNG, PDF)
\usepackage{caption}
\usepackage{subcaption}
\usepackage{multirow}                		% Fletning af raekker og kolonner (\multicolumn og \multirow)
\usepackage{makecell}                       % Line breaks i tabelceller med \makecell{bla bla \\ bla bla}
\usepackage{colortbl} 						% Farver i tabeller (fx \columncolor, \rowcolor og \cellcolor)
\usepackage[dvipsnames,table,longtable,x11names]{xcolor}

%Tabel automatisk ny linje
\usepackage{array}
\newcolumntype{L}[1]{>{\raggedright\let\newline\\\arraybackslash\hspace{0pt}}m{#1}}
\newcolumntype{C}[1]{>{\centering\let\newline\\\arraybackslash\hspace{0pt}}m{#1}}
\newcolumntype{R}[1]{>{\raggedleft\let\newline\\\arraybackslash\hspace{0pt}}m{#1}}

% Definer farver med \definecolor. Se mere: http://en.wikibooks.org/wiki/LaTeX/Colors
\usepackage{flafter}						% Soerger for at floats ikke optraeder i teksten foer deres reference
\let\newfloat\relax 						% Justering mellem float-pakken og memoir
\usepackage{float}							% Muliggoer eksakt placering af floats, f.eks.
\usepackage{afterpage}
%\usepackage{scrextend}                      % labeling lister
\usepackage{chngcntr} % kontroller nummerering af floats
\counterwithin{figure}{section} % sæt nummerering efter sektion
\counterwithin{table}{section} % sæt nummerering efter sektion

% ¤¤ Matematik mm. ¤¤
\usepackage{amsmath,amssymb,stmaryrd} 		% Avancerede matematik-udvidelser
\usepackage{mathtools}						% Andre matematik- og tegnudvidelser
\usepackage{textcomp}                 		% Symbol-udvidelser (f.eks. promille-tegn med \textperthousand )
\usepackage{siunitx}						% Flot og konsistent praesentation af tal og enheder med \si{enhed} og \SI{tal}{enhed}
\sisetup{output-decimal-marker = {,}}		% Opsaetning af \SI (DE for komma som decimalseparator)

% ¤¤ Misc. ¤¤ %
\usepackage{listings}						% Placer kildekode i dokumentet med \begin{lstlisting}...\end{lstlisting}
\usepackage{blindtext}
\usepackage{lipsum}							% Dummy text \lipsum[..]
\usepackage[shortlabels]{enumitem}			% Muliggoer enkelt konfiguration af lister
\usepackage{pdfpages}						% Goer det muligt at inkludere pdf-dokumenter med kommandoen \includepdf[pages={x-y}]{fil.pdf}
\usepackage[bottom]{footmisc}               % Saetter footnotes i bunden af siden
\pdfoptionpdfminorversion=6					% Muliggoer inkludering af pdf dokumenter, af version 1.6 og hoejere
\pretolerance=2500 							% Justering af afstand mellem ord (hoejt tal, mindre orddeling og mere luft mellem ord)


%%%% BRUGERDEFINEREDE INDSTILLINGER %%%%

\linespread{1,1}							% Linie afstand

% ¤¤ Visuelle referencer ¤¤ %
\usepackage[colorlinks,pdfencoding=auto]{hyperref}			% Danner klikbare referencer (hyperlinks) i dokumentet.
\hypersetup{colorlinks = true,				% Opsaetning af farvede hyperlinks (interne links, citeringer og URL)
    linkcolor = black,
    citecolor = black,
    urlcolor = black
}

%%%% TODO-NOTER %%%%
\usepackage[danish, colorinlistoftodos]{todonotes}
%\usepackage[colorinlistoftodos]{todonotes}

%%%% TABEL BAGGRUNDSFARVER %%%%
\definecolor{aublueclassic}{RGB}{0,61,115}
\definecolor{aubluedark}{RGB}{0,37,70}
\definecolor{aucyan}{RGB}{225,248,253}
%\definecolor{aucyan}{RGB}{55,160,203}
\definecolor{aucyandark}{RGB}{0,62,92}
\definecolor{lightGray}{RGB}{153,153,153}
\definecolor{darkGray}{RGB}{119,119,119}
\definecolor{khaki}{RGB}{240,230,140}
\definecolor{lavender}{RGB}{230,230,250}

%indentering af section
\usepackage{changepage}

%Code highlighting
\usepackage{minted}

%%%% Tabs %%%%
\usepackage{tabto}
\NumTabs{10}

%%%% REFERENCE TIL SECTION-NAME %%%%
\usepackage{nameref}

%sidehoved
\usepackage{fancyhdr}
\usepackage{lastpage}

%referer til afsnit i andre dokumenter
\usepackage{xr}
\makeatletter
\newcommand*{\addFileDependency}[1]{% argument=file name and extension
  \typeout{(#1)}
  \@addtofilelist{#1}
  \IfFileExists{#1}{}{\typeout{No file #1.}}
}
\makeatother
 
\newcommand*{\myexternaldocument}[1]{%
    \externaldocument{#1}%
    \addFileDependency{#1.tex}%
    \addFileDependency{#1.aux}%
}

%%%% paragraph %%%%
%\usepackage{titlesec}
%\setcounter{secnumdepth}{4}

%\titleformat{\paragraph}
%{\normalfont\normalsize\bfseries}{\theparagraph}{1em}{}
%\titlespacing*{\paragraph}
%{0pt}{3.25ex plus 1ex minus .2ex}{1.5ex plus .2ex}

% ========== PAKKER DER SKAL LOADES TIL SIDST ==================
%\usepackage{xcolor}
%\usepackage{listings}
\usepackage{csquotes}           %så holder bilatex kæft
\usepackage{subfiles}


%%EDS SHIT
\PassOptionsToPackage{unicode=true}{hyperref} % options for packages loaded elsewhere
\PassOptionsToPackage{hyphens}{url}
%
\usepackage{lmodern}
\usepackage{amssymb,amsmath}
\usepackage{ifxetex,ifluatex}
\ifnum 0\ifxetex 1\fi\ifluatex 1\fi=0 % if pdftex
  \usepackage[T1]{fontenc}
  \usepackage[utf8]{inputenc}
  \usepackage{textcomp} % provides euro and other symbols
\else % if luatex or xelatex
  \usepackage{unicode-math}
  \defaultfontfeatures{Ligatures=TeX,Scale=MatchLowercase}
\fi
% use upquote if available, for straight quotes in verbatim environments
\IfFileExists{upquote.sty}{\usepackage{upquote}}{}
% use microtype if available
\IfFileExists{microtype.sty}{%
\usepackage[]{microtype}
\UseMicrotypeSet[protrusion]{basicmath} % disable protrusion for tt fonts
}{}
\IfFileExists{parskip.sty}{%
\usepackage{parskip}
}{% else
\setlength{\parindent}{0pt}
\setlength{\parskip}{6pt plus 2pt minus 1pt}
}
\usepackage{hyperref}
\hypersetup{
            pdfborder={0 0 0},
            breaklinks=true}
\urlstyle{same}  % don't use monospace font for urls
\usepackage{longtable,booktabs}
% Fix footnotes in tables (requires footnote package)
\IfFileExists{footnote.sty}{\usepackage{footnote}\makesavenoteenv{longtable}}{}
\setlength{\emergencystretch}{3em}  % prevent overfull lines
\providecommand{\tightlist}{%
  \setlength{\itemsep}{0pt}\setlength{\parskip}{0pt}}
% Redefines (sub)paragraphs to behave more like sections
\ifx\paragraph\undefined\else
\let\oldparagraph\paragraph
\renewcommand{\paragraph}[1]{\oldparagraph{#1}\mbox{}}
\fi
\ifx\subparagraph\undefined\else
\let\oldsubparagraph\subparagraph
\renewcommand{\subparagraph}[1]{\oldsubparagraph{#1}\mbox{}}
\fi

% set default figure placement to htbp
\makeatletter
\def\fps@figure{htbp}
\makeatother